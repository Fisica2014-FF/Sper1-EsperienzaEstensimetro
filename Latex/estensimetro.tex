\documentclass[12pt]{article} % Prepara un documento con un font grande
\usepackage[italian]{babel} % Adatta LaTeX alle convenzioni tipografiche italiane,
							% e ridefinisce alcuni titoli in italiano, come "Capitolo" al posto di "Chapter",
							% se il documento è in italiano

\usepackage[utf8]{inputenc} % Consente l'uso caratteri accentati italiani
\usepackage{graphicx}		% Per le immagini
\usepackage{float}
\usepackage[top=2in, bottom=1.5in, left=0.5in, right=0.5in]{geometry}

\nonstopmode %non fermarti agli errori

\title {Relazione di Laboratorio - Estensimetro}
\author{Francesco Forcher\\
Facoltà di Fisica\\
Università di Padova\\
Matricola: \texttt{1073458}\\
\texttt{mailto:francesco.forcher@studenti.unipd.it}\\
\and
Andrea Piccinin\\ 
Facoltà di Fisica\\
Università di Padova\\
Matricola: \texttt{1070620}\\
\texttt{mailto:andrea.piccinin1@studenti.unipd.it}\\
}

\date{\today}


\pagestyle{headings}
\DeclareGraphicsExtensions{.pdf, .png, .jpg} % Se due immagini hanno lo stesso nome sceglile secondo l'ordine di filetype qui

\graphicspath{ {./img/} }%%%%%%%%%%%%%%%%%%%%%%%%%%%%%%%%EDITARE PATH!!%%%%%%%%%%%%%%%%%%%%%%%%%%%%%%%%%%%%%%%%% % Path delle immagini 


% PROVA MODIFICA DA BROWSER!!!!
%////////////////////////////////////////////////////////////////////////////////////////////////////////////////////////////
%////////////////////////////////////////////////////////////////////////////////////////////////////////////////////////////
% Fine dei dati iniziali per il latex: il documento finale inizierà da qui
\begin{document}


\maketitle % Produce il titolo a partire dai comandi \title, \author e \date
\tableofcontents % Prepara l'indice generale


\section{Obiettivi}
	L'obiettivo dell'esperienza è quello di calcolare, tramite diversi valori della costante elastica K presa per diversi fili metallici, il valore del modulo di Young, E, di tre diversi materiali: Acciaio, Ottone e Tungsteno.		
\section{Descrizione dell'apparato strumentale}
	Gli apparati sperimentali utilizzati durante quest'esperienza sono stati diversi estensimetri, ognuno dotato di un diverso filo metallico (fili d'acciao, ottone e tungsteno) e diverse caratteristiche geometriche (diametro, d, e lunghezza iniziale, x\ped{0}).
Ogni estensimetro è dotato di un sistema meccanico manuale in grado di applicare una tensione al filo; Questo sistema è dotato di un indicatore (sensibilità indicatore forza = 100 g\ped{f}) con la particolarità che la forza effettivamente applicata al filo è quattro volte la forza letta su di esso: F\ped{applicata} = 4*F\ped{letta}.
L'estensimetro è inoltre dotato di un contagiri, collegato al filo metellico, in grado di misurarne l'allungamento; La sensibilità di questo apparecchio è di 1$\mu$m.	
\section{Metodologia di misura}
				
\section{Presentazione dati sperimentali ed elaborazione dati}			
	
	\subsection {Inclinazione 15', senza peso}

\section{Discussione dati sperimentali}

\section{Conclusioni}
	
% Esempio di inclusione di un immagine ./img/spazio1.png etichettata fig:spazio, al centro, larga 0.8 per larghezza del testo, con testo sotto Spazio!
\section {Grafici}
	\subsection {}

	%\include{}
	
\section{Codice}	

\paragraph{}
	
%\subsection{Esempio immagini}
%\begin{figure}[p]
%    \centering
%    \includegraphics[width=0.8\textwidth]{spazio1}
%    \caption{Spazio!}
%    \label{fig:spazio1}
%\end{figure}

\end{document}
