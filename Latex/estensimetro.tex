\documentclass[12pt]{article} % Prepara un documento con un font grande
\usepackage[italian]{babel} % Adatta LaTeX alle convenzioni tipografiche italiane,
							% e ridefinisce alcuni titoli in italiano, come "Capitolo" al posto di "Chapter",
							% se il documento è in italiano

\usepackage[utf8]{inputenc} % Consente l'uso caratteri accentati italiani
\usepackage{graphicx}		% Per le immagini
\usepackage{float}
\usepackage[top=2in, bottom=1.5in, left=0.5in, right=0.5in]{geometry}

\nonstopmode %non fermarti agli errori

\title {Relazione di Laboratorio - Estensimetro}
\author{Francesco Forcher\\
Facoltà di Fisica\\
Università di Padova\\
Matricola: \texttt{1073458}\\
\texttt{mailto:francesco.forcher@studenti.unipd.it}\\
\and
Andrea Piccinin\\ 
Facoltà di Fisica\\
Università di Padova\\
Matricola: \texttt{1070620}\\
\texttt{mailto:andrea.piccinin1@studenti.unipd.it}\\
}

\date{\today}


\pagestyle{headings}
\DeclareGraphicsExtensions{.pdf, .png, .jpg} % Se due immagini hanno lo stesso nome sceglile secondo l'ordine di filetype qui

\graphicspath{ {./img/} }%%%%%%%%%%%%%%%%%%%%%%%%%%%%%%%%EDITARE PATH!!%%%%%%%%%%%%%%%%%%%%%%%%%%%%%%%%%%%%%%%%% % Path delle immagini 















% PROVA MODIFICA DA BROWSER!!!!
%////////////////////////////////////////////////////////////////////////////////////////////////////////////////////////////
%////////////////////////////////////////////////////////////////////////////////////////////////////////////////////////////
% Fine dei dati iniziali per il latex: il documento finale inizierà da qui
\begin{document}


\maketitle % Produce il titolo a partire dai comandi \title, \author e \date
\tableofcontents % Prepara l'indice generale



        

\section{Obiettivi}
		
\section{Descrizione dell'apparato strumentale}
		
\section{Metodologia di misura}
		
\section{Presentazione dati sperimentali}			
	Riportiamo in seguito le misure tabulate, con relative statistiche.
	
	\subsection {Inclinazione 15', senza peso}
	La quarta misura nell'intervallo 40-60 cm ha un valore non compatibile con gli altri. Potrebbe esserci stato un errore nella trascrizione dei dati durante la misurazione.
	%\include{}

\section{Discussione dati sperimentali}
	Compatibilmente con le previsioni teoriche, si è verificato che l'accelerazione della slitta non dipende dalla sua massa, infatti, 		aggiungendo il peso, varia soltanto la velocità, e che, diminuendo la velocità iniziale della slitta con l'aggiunta dello spessore, 		la decelerazione data dalla forza di attrito dell'aria è rimasta invariata. 

\section{Conclusioni}
	Si ha una migliore stima di \textbf{g} 
	
% Esempio di inclusione di un immagine ./img/spazio1.png etichettata fig:spazio, al centro, larga 0.8 per larghezza del testo, con testo sotto Spazio!
\section {Grafici}
	Di seguito sono riportati i grafici delle velocità medie e le rette interpolanti.
	\subsection {}
	%\include{}
	
\section{Codice}	

\paragraph{}
	Riportiamo in seguito il programma utilizzato per l'elaborazione dei dati
	%Verbatim non interpreta l'imput lasciando il testo com'è: ideale per inserire codice
	\begin{verbatim}

	\end{verbatim}


%\subsection{Esempio immagini}
%\begin{figure}[p]
%    \centering
%    \includegraphics[width=0.8\textwidth]{spazio1}
%    \caption{Spazio!}
%    \label{fig:spazio1}
%\end{figure}

\end{document}
