\documentclass[12pt]{article} % Prepara un documento con un font grande
\usepackage[italian]{babel} % Adatta LaTeX alle convenzioni tipografiche italiane,
% e ridefinisce alcuni titoli in italiano, come "Capitolo" al posto di "Chapter",
% se il documento è in italiano

\usepackage[utf8]{inputenc} % Consente l'uso caratteri accentati italiani
\usepackage{graphicx}	% Per le immagini
\usepackage{float}
%\usepackage{fancyhdr}
\usepackage{gnuplot-lua-tikz}
\usepackage[top=1.5cm, bottom=1.5cm, left=1.5cm, right=1.5cm]{geometry}

\nonstopmode %non fermarti agli errori

\title {Relazione di Laboratorio - Estensimetro}
\author{Francesco Forcher\\
Facoltà di Fisica\\
Università di Padova\\
Matricola: \texttt{1073458}\\
\texttt{mailto:francesco.forcher@studenti.unipd.it}\\
\and
Andrea Piccinin\\
Facoltà di Fisica\\
Università di Padova\\
Matricola: \texttt{1070620}\\
\texttt{mailto:andrea.piccinin1@studenti.unipd.it}\\
}

\date{\today}


\pagestyle{headings}
\DeclareGraphicsExtensions{.pdf, .png, .jpg} % Se due immagini hanno lo stesso nome sceglile secondo l'ordine di filetype qui

\graphicspath{ {./img/} }%%%%%%%%%%%%%%%%%%%%%%%%%%%%%%%%EDITARE PATH!!%%%%%%%%%%%%%%%%%%%%%%%%%%%%%%%%%%%%%%%%% % Path delle immagini






%%%%%%%%%%%%%%%%%%%%%%%%%%%%%%%%%%%%%%5%%%%%%%%%%%%%%%%%%%%%%%%%%%%%%%%%%%%%%%%%%%
%\usepackage{float}
\usepackage{caption}
%\usepackage{multirow}
%\usepackage[top=3.6cm, bottom=1.5in, left=0.5in, right=0.5in]{geometry}

\nonstopmode

\floatstyle{ruled}
\newfloat{tabella}{H}{lop}
\floatname{tabella}{Tabella}

\floatstyle{ruled}
\newfloat{grafico}{H}{lop}
\floatname{grafico}{Grafico}
%%%%%%%%%%%%%%%%%%%%%%%%%%%%%%%%%%%%%%%%%%%%%%%%%%%%%%%%%%%%%%%%%%%%%5%%%%%%%%%%%%%







% PROVA MODIFICA DA BROWSER!!!!
%////////////////////////////////////////////////////////////////////////////////////////////////////////////////////////////
%////////////////////////////////////////////////////////////////////////////////////////////////////////////////////////////
% Fine dei dati iniziali per il latex: il documento finale inizierà da qui
\begin{document}


\maketitle % Produce il titolo a partire dai comandi \title, \author e \date
\tableofcontents % Prepara l'indice generale



        

\section{Obiettivi}

\section{Descrizione dell'apparato strumentale}

\section{Metodologia di misura}

\section{Presentazione dati sperimentali}	
Riportiamo in seguito le misure tabulate, con relative statistiche.

\subsection {Inclinazione 15', senza peso}
La quarta misura nell'intervallo 40-60 cm ha un valore non compatibile con gli altri. Potrebbe esserci stato un errore nella trascrizione dei dati durante la misurazione.
\begin{grafico}
    \centering
\begin{tikzpicture}[gnuplot]
%% generated with GNUPLOT 4.6p0 (Lua 5.1; terminal rev. 99, script rev. 100)
%% Tue 25 Mar 2014 05:11:20 PM CET
\path (0.000,0.000) rectangle (12.500,8.750);
\gpcolor{color=gp lt color axes}
\gpsetlinetype{gp lt axes}
\gpsetlinewidth{1.00}
\draw[gp path] (1.504,0.985)--(11.947,0.985);
\gpcolor{color=gp lt color border}
\gpsetlinetype{gp lt border}
\draw[gp path] (1.504,0.985)--(1.684,0.985);
\draw[gp path] (11.947,0.985)--(11.767,0.985);
\node[gp node right] at (1.320,0.985) { 0};
\gpcolor{color=gp lt color axes}
\gpsetlinetype{gp lt axes}
\draw[gp path] (1.504,2.353)--(8.271,2.353);
\draw[gp path] (11.763,2.353)--(11.947,2.353);
\gpcolor{color=gp lt color border}
\gpsetlinetype{gp lt border}
\draw[gp path] (1.504,2.353)--(1.684,2.353);
\draw[gp path] (11.947,2.353)--(11.767,2.353);
\node[gp node right] at (1.320,2.353) { 50};
\gpcolor{color=gp lt color axes}
\gpsetlinetype{gp lt axes}
\draw[gp path] (1.504,3.721)--(11.947,3.721);
\gpcolor{color=gp lt color border}
\gpsetlinetype{gp lt border}
\draw[gp path] (1.504,3.721)--(1.684,3.721);
\draw[gp path] (11.947,3.721)--(11.767,3.721);
\node[gp node right] at (1.320,3.721) { 100};
\gpcolor{color=gp lt color axes}
\gpsetlinetype{gp lt axes}
\draw[gp path] (1.504,5.089)--(11.947,5.089);
\gpcolor{color=gp lt color border}
\gpsetlinetype{gp lt border}
\draw[gp path] (1.504,5.089)--(1.684,5.089);
\draw[gp path] (11.947,5.089)--(11.767,5.089);
\node[gp node right] at (1.320,5.089) { 150};
\gpcolor{color=gp lt color axes}
\gpsetlinetype{gp lt axes}
\draw[gp path] (1.504,6.457)--(11.947,6.457);
\gpcolor{color=gp lt color border}
\gpsetlinetype{gp lt border}
\draw[gp path] (1.504,6.457)--(1.684,6.457);
\draw[gp path] (11.947,6.457)--(11.767,6.457);
\node[gp node right] at (1.320,6.457) { 200};
\gpcolor{color=gp lt color axes}
\gpsetlinetype{gp lt axes}
\draw[gp path] (1.504,7.825)--(11.947,7.825);
\gpcolor{color=gp lt color border}
\gpsetlinetype{gp lt border}
\draw[gp path] (1.504,7.825)--(1.684,7.825);
\draw[gp path] (11.947,7.825)--(11.767,7.825);
\node[gp node right] at (1.320,7.825) { 250};
\gpcolor{color=gp lt color axes}
\gpsetlinetype{gp lt axes}
\draw[gp path] (1.504,0.985)--(1.504,7.825);
\gpcolor{color=gp lt color border}
\gpsetlinetype{gp lt border}
\draw[gp path] (1.504,0.985)--(1.504,1.165);
\draw[gp path] (1.504,7.825)--(1.504,7.645);
\node[gp node center] at (1.504,0.677) { 0};
\gpcolor{color=gp lt color axes}
\gpsetlinetype{gp lt axes}
\draw[gp path] (2.664,0.985)--(2.664,7.825);
\gpcolor{color=gp lt color border}
\gpsetlinetype{gp lt border}
\draw[gp path] (2.664,0.985)--(2.664,1.165);
\draw[gp path] (2.664,7.825)--(2.664,7.645);
\node[gp node center] at (2.664,0.677) { 5};
\gpcolor{color=gp lt color axes}
\gpsetlinetype{gp lt axes}
\draw[gp path] (3.825,0.985)--(3.825,7.825);
\gpcolor{color=gp lt color border}
\gpsetlinetype{gp lt border}
\draw[gp path] (3.825,0.985)--(3.825,1.165);
\draw[gp path] (3.825,7.825)--(3.825,7.645);
\node[gp node center] at (3.825,0.677) { 10};
\gpcolor{color=gp lt color axes}
\gpsetlinetype{gp lt axes}
\draw[gp path] (4.985,0.985)--(4.985,7.825);
\gpcolor{color=gp lt color border}
\gpsetlinetype{gp lt border}
\draw[gp path] (4.985,0.985)--(4.985,1.165);
\draw[gp path] (4.985,7.825)--(4.985,7.645);
\node[gp node center] at (4.985,0.677) { 15};
\gpcolor{color=gp lt color axes}
\gpsetlinetype{gp lt axes}
\draw[gp path] (6.145,0.985)--(6.145,7.825);
\gpcolor{color=gp lt color border}
\gpsetlinetype{gp lt border}
\draw[gp path] (6.145,0.985)--(6.145,1.165);
\draw[gp path] (6.145,7.825)--(6.145,7.645);
\node[gp node center] at (6.145,0.677) { 20};
\gpcolor{color=gp lt color axes}
\gpsetlinetype{gp lt axes}
\draw[gp path] (7.306,0.985)--(7.306,7.825);
\gpcolor{color=gp lt color border}
\gpsetlinetype{gp lt border}
\draw[gp path] (7.306,0.985)--(7.306,1.165);
\draw[gp path] (7.306,7.825)--(7.306,7.645);
\node[gp node center] at (7.306,0.677) { 25};
\gpcolor{color=gp lt color axes}
\gpsetlinetype{gp lt axes}
\draw[gp path] (8.466,0.985)--(8.466,1.165);
\draw[gp path] (8.466,2.397)--(8.466,7.825);
\gpcolor{color=gp lt color border}
\gpsetlinetype{gp lt border}
\draw[gp path] (8.466,0.985)--(8.466,1.165);
\draw[gp path] (8.466,7.825)--(8.466,7.645);
\node[gp node center] at (8.466,0.677) { 30};
\gpcolor{color=gp lt color axes}
\gpsetlinetype{gp lt axes}
\draw[gp path] (9.626,0.985)--(9.626,1.165);
\draw[gp path] (9.626,2.397)--(9.626,7.825);
\gpcolor{color=gp lt color border}
\gpsetlinetype{gp lt border}
\draw[gp path] (9.626,0.985)--(9.626,1.165);
\draw[gp path] (9.626,7.825)--(9.626,7.645);
\node[gp node center] at (9.626,0.677) { 35};
\gpcolor{color=gp lt color axes}
\gpsetlinetype{gp lt axes}
\draw[gp path] (10.787,0.985)--(10.787,1.165);
\draw[gp path] (10.787,2.397)--(10.787,7.825);
\gpcolor{color=gp lt color border}
\gpsetlinetype{gp lt border}
\draw[gp path] (10.787,0.985)--(10.787,1.165);
\draw[gp path] (10.787,7.825)--(10.787,7.645);
\node[gp node center] at (10.787,0.677) { 40};
\gpcolor{color=gp lt color axes}
\gpsetlinetype{gp lt axes}
\draw[gp path] (11.947,0.985)--(11.947,7.825);
\gpcolor{color=gp lt color border}
\gpsetlinetype{gp lt border}
\draw[gp path] (11.947,0.985)--(11.947,1.165);
\draw[gp path] (11.947,7.825)--(11.947,7.645);
\node[gp node center] at (11.947,0.677) { 45};
\draw[gp path] (1.504,7.825)--(1.504,0.985)--(11.947,0.985)--(11.947,7.825)--cycle;
\node[gp node center,rotate=-270] at (0.246,4.405) {Allungamento ($10^{-5} [m]$)};
\node[gp node center] at (6.725,0.215) {Forza (N)};
\node[gp node center] at (6.725,8.287) {Distribuzione dei dati: E3};
\node[gp node left] at (8.271,2.243) {Dati andata};
\gpcolor{color=gp lt color 0}
\gpsetlinetype{gp lt plot 0}
\draw[gp path] (10.663,2.243)--(11.579,2.243);
\draw[gp path] (10.663,2.333)--(10.663,2.153);
\draw[gp path] (11.579,2.333)--(11.579,2.153);
\draw[gp path] (2.319,0.985)--(2.499,0.985);
\draw[gp path] (2.319,0.985)--(2.499,0.985);
\draw[gp path] (3.224,1.560)--(3.404,1.560);
\draw[gp path] (3.224,1.560)--(3.404,1.560);
\draw[gp path] (4.152,2.134)--(4.332,2.134);
\draw[gp path] (4.152,2.134)--(4.332,2.134);
\draw[gp path] (5.057,2.681)--(5.237,2.681);
\draw[gp path] (5.057,2.681)--(5.237,2.681);
\draw[gp path] (5.963,3.242)--(6.143,3.242);
\draw[gp path] (5.963,3.242)--(6.143,3.242);
\draw[gp path] (6.868,3.776)--(7.048,3.776);
\draw[gp path] (6.868,3.776)--(7.048,3.776);
\draw[gp path] (7.773,4.350)--(7.953,4.350);
\draw[gp path] (7.773,4.350)--(7.953,4.350);
\draw[gp path] (8.701,4.925)--(8.881,4.925);
\draw[gp path] (8.701,4.925)--(8.881,4.925);
\draw[gp path] (9.606,5.472)--(9.786,5.472);
\draw[gp path] (9.606,5.472)--(9.786,5.472);
\draw[gp path] (10.511,5.992)--(10.691,5.992);
\draw[gp path] (10.511,5.992)--(10.691,5.992);
\draw[gp path] (11.416,6.553)--(11.596,6.553);
\draw[gp path] (11.416,6.553)--(11.596,6.553);
\gpsetpointsize{4.00}
\gppoint{gp mark 1}{(2.409,0.985)}
\gppoint{gp mark 1}{(3.314,1.560)}
\gppoint{gp mark 1}{(4.242,2.134)}
\gppoint{gp mark 1}{(5.147,2.681)}
\gppoint{gp mark 1}{(6.053,3.242)}
\gppoint{gp mark 1}{(6.958,3.776)}
\gppoint{gp mark 1}{(7.863,4.350)}
\gppoint{gp mark 1}{(8.791,4.925)}
\gppoint{gp mark 1}{(9.696,5.472)}
\gppoint{gp mark 1}{(10.601,5.992)}
\gppoint{gp mark 1}{(11.506,6.553)}
\gppoint{gp mark 1}{(11.121,2.243)}
\gpcolor{color=gp lt color border}
\node[gp node left] at (8.271,1.935) {Dati ritorno};
\gpcolor{color=gp lt color 1}
\gpsetlinetype{gp lt plot 1}
\draw[gp path] (10.663,1.935)--(11.579,1.935);
\draw[gp path] (10.663,2.025)--(10.663,1.845);
\draw[gp path] (11.579,2.025)--(11.579,1.845);
\draw[gp path] (11.416,6.553)--(11.596,6.553);
\draw[gp path] (11.416,6.553)--(11.596,6.553);
\draw[gp path] (10.511,6.019)--(10.691,6.019);
\draw[gp path] (10.511,6.019)--(10.691,6.019);
\draw[gp path] (9.606,5.472)--(9.786,5.472);
\draw[gp path] (9.606,5.472)--(9.786,5.472);
\draw[gp path] (8.701,4.897)--(8.881,4.897);
\draw[gp path] (8.701,4.897)--(8.881,4.897);
\draw[gp path] (7.773,4.323)--(7.953,4.323);
\draw[gp path] (7.773,4.323)--(7.953,4.323);
\draw[gp path] (6.868,3.748)--(7.048,3.748);
\draw[gp path] (6.868,3.748)--(7.048,3.748);
\draw[gp path] (5.963,3.201)--(6.143,3.201);
\draw[gp path] (5.963,3.201)--(6.143,3.201);
\draw[gp path] (5.057,2.681)--(5.237,2.681);
\draw[gp path] (5.057,2.681)--(5.237,2.681);
\draw[gp path] (4.152,2.107)--(4.332,2.107);
\draw[gp path] (4.152,2.107)--(4.332,2.107);
\draw[gp path] (3.224,1.560)--(3.404,1.560);
\draw[gp path] (3.224,1.560)--(3.404,1.560);
\draw[gp path] (2.319,0.999)--(2.499,0.999);
\draw[gp path] (2.319,0.999)--(2.499,0.999);
\gppoint{gp mark 2}{(11.506,6.553)}
\gppoint{gp mark 2}{(10.601,6.019)}
\gppoint{gp mark 2}{(9.696,5.472)}
\gppoint{gp mark 2}{(8.791,4.897)}
\gppoint{gp mark 2}{(7.863,4.323)}
\gppoint{gp mark 2}{(6.958,3.748)}
\gppoint{gp mark 2}{(6.053,3.201)}
\gppoint{gp mark 2}{(5.147,2.681)}
\gppoint{gp mark 2}{(4.242,2.107)}
\gppoint{gp mark 2}{(3.314,1.560)}
\gppoint{gp mark 2}{(2.409,0.999)}
\gppoint{gp mark 2}{(11.121,1.935)}
\gpcolor{color=gp lt color border}
\node[gp node left] at (8.271,1.627) {Andata};
\gpcolor{color=gp lt color 2}
\gpsetlinetype{gp lt plot 2}
\draw[gp path] (10.663,1.627)--(11.579,1.627);
\draw[gp path] (2.409,1.005)--(2.501,1.062)--(2.593,1.118)--(2.685,1.174)--(2.777,1.230)%
  --(2.869,1.286)--(2.960,1.342)--(3.052,1.399)--(3.144,1.455)--(3.236,1.511)--(3.328,1.567)%
  --(3.420,1.623)--(3.512,1.679)--(3.604,1.736)--(3.696,1.792)--(3.787,1.848)--(3.879,1.904)%
  --(3.971,1.960)--(4.063,2.016)--(4.155,2.073)--(4.247,2.129)--(4.339,2.185)--(4.431,2.241)%
  --(4.523,2.297)--(4.614,2.353)--(4.706,2.410)--(4.798,2.466)--(4.890,2.522)--(4.982,2.578)%
  --(5.074,2.634)--(5.166,2.690)--(5.258,2.747)--(5.350,2.803)--(5.441,2.859)--(5.533,2.915)%
  --(5.625,2.971)--(5.717,3.027)--(5.809,3.084)--(5.901,3.140)--(5.993,3.196)--(6.085,3.252)%
  --(6.177,3.308)--(6.268,3.364)--(6.360,3.421)--(6.452,3.477)--(6.544,3.533)--(6.636,3.589)%
  --(6.728,3.645)--(6.820,3.701)--(6.912,3.757)--(7.004,3.814)--(7.095,3.870)--(7.187,3.926)%
  --(7.279,3.982)--(7.371,4.038)--(7.463,4.094)--(7.555,4.151)--(7.647,4.207)--(7.739,4.263)%
  --(7.831,4.319)--(7.922,4.375)--(8.014,4.431)--(8.106,4.488)--(8.198,4.544)--(8.290,4.600)%
  --(8.382,4.656)--(8.474,4.712)--(8.566,4.768)--(8.658,4.825)--(8.749,4.881)--(8.841,4.937)%
  --(8.933,4.993)--(9.025,5.049)--(9.117,5.105)--(9.209,5.162)--(9.301,5.218)--(9.393,5.274)%
  --(9.485,5.330)--(9.576,5.386)--(9.668,5.442)--(9.760,5.499)--(9.852,5.555)--(9.944,5.611)%
  --(10.036,5.667)--(10.128,5.723)--(10.220,5.779)--(10.312,5.836)--(10.403,5.892)--(10.495,5.948)%
  --(10.587,6.004)--(10.679,6.060)--(10.771,6.116)--(10.863,6.173)--(10.955,6.229)--(11.047,6.285)%
  --(11.139,6.341)--(11.230,6.397)--(11.322,6.453)--(11.414,6.510)--(11.506,6.566);
\gpcolor{color=gp lt color border}
\node[gp node left] at (8.271,1.319) {Ritorno};
\gpcolor{color=gp lt color 3}
\gpsetlinetype{gp lt plot 3}
\draw[gp path] (10.663,1.319)--(11.579,1.319);
\draw[gp path] (2.409,0.992)--(2.501,1.048)--(2.593,1.104)--(2.685,1.160)--(2.777,1.217)%
  --(2.869,1.273)--(2.960,1.329)--(3.052,1.385)--(3.144,1.442)--(3.236,1.498)--(3.328,1.554)%
  --(3.420,1.610)--(3.512,1.667)--(3.604,1.723)--(3.696,1.779)--(3.787,1.835)--(3.879,1.892)%
  --(3.971,1.948)--(4.063,2.004)--(4.155,2.060)--(4.247,2.117)--(4.339,2.173)--(4.431,2.229)%
  --(4.523,2.285)--(4.614,2.342)--(4.706,2.398)--(4.798,2.454)--(4.890,2.510)--(4.982,2.566)%
  --(5.074,2.623)--(5.166,2.679)--(5.258,2.735)--(5.350,2.791)--(5.441,2.848)--(5.533,2.904)%
  --(5.625,2.960)--(5.717,3.016)--(5.809,3.073)--(5.901,3.129)--(5.993,3.185)--(6.085,3.241)%
  --(6.177,3.298)--(6.268,3.354)--(6.360,3.410)--(6.452,3.466)--(6.544,3.523)--(6.636,3.579)%
  --(6.728,3.635)--(6.820,3.691)--(6.912,3.748)--(7.004,3.804)--(7.095,3.860)--(7.187,3.916)%
  --(7.279,3.972)--(7.371,4.029)--(7.463,4.085)--(7.555,4.141)--(7.647,4.197)--(7.739,4.254)%
  --(7.831,4.310)--(7.922,4.366)--(8.014,4.422)--(8.106,4.479)--(8.198,4.535)--(8.290,4.591)%
  --(8.382,4.647)--(8.474,4.704)--(8.566,4.760)--(8.658,4.816)--(8.749,4.872)--(8.841,4.929)%
  --(8.933,4.985)--(9.025,5.041)--(9.117,5.097)--(9.209,5.153)--(9.301,5.210)--(9.393,5.266)%
  --(9.485,5.322)--(9.576,5.378)--(9.668,5.435)--(9.760,5.491)--(9.852,5.547)--(9.944,5.603)%
  --(10.036,5.660)--(10.128,5.716)--(10.220,5.772)--(10.312,5.828)--(10.403,5.885)--(10.495,5.941)%
  --(10.587,5.997)--(10.679,6.053)--(10.771,6.110)--(10.863,6.166)--(10.955,6.222)--(11.047,6.278)%
  --(11.139,6.335)--(11.230,6.391)--(11.322,6.447)--(11.414,6.503)--(11.506,6.559);
\gpcolor{color=gp lt color border}
\gpsetlinetype{gp lt border}
\draw[gp path] (1.504,7.825)--(1.504,0.985)--(11.947,0.985)--(11.947,7.825)--cycle;
%% coordinates of the plot area
\gpdefrectangularnode{gp plot 1}{\pgfpoint{1.504cm}{0.985cm}}{\pgfpoint{11.947cm}{7.825cm}}
\end{tikzpicture}
%% gnuplot variables

\caption{Dati del primo estensimetro}
\label{fig:spazio1}
\end{grafico}
Dati una serie di file di dati .fdat, voglio un sistema coerente e consistente per effettuare operazioni
identiche (nel senso che la variabilità dipende solo da info nei file di template o nei dati) su di essi, dove
con operazione si intende una funzione F : (file fdat, template) => file di output che crea un file dove
le variabili di template, indicate COSI, sono sostituite con il loro valore hash corrispondente:


\begin{figure}[H]
    \centering
\begin{tikzpicture}[gnuplot]
%% generated with GNUPLOT 4.6p0 (Lua 5.1; terminal rev. 99, script rev. 100)
%% Tue 25 Mar 2014 10:37:29 AM CET
\path (0.000,0.000) rectangle (12.500,8.750);
\gpcolor{color=gp lt color axes}
\gpsetlinetype{gp lt axes}
\gpsetlinewidth{1.00}
\draw[gp path] (1.504,0.985)--(11.947,0.985);
\gpcolor{color=gp lt color border}
\gpsetlinetype{gp lt border}
\draw[gp path] (1.504,0.985)--(1.684,0.985);
\draw[gp path] (11.947,0.985)--(11.767,0.985);
\node[gp node right] at (1.320,0.985) { 0};
\gpcolor{color=gp lt color axes}
\gpsetlinetype{gp lt axes}
\draw[gp path] (1.504,2.353)--(8.271,2.353);
\draw[gp path] (11.763,2.353)--(11.947,2.353);
\gpcolor{color=gp lt color border}
\gpsetlinetype{gp lt border}
\draw[gp path] (1.504,2.353)--(1.684,2.353);
\draw[gp path] (11.947,2.353)--(11.767,2.353);
\node[gp node right] at (1.320,2.353) { 50};
\gpcolor{color=gp lt color axes}
\gpsetlinetype{gp lt axes}
\draw[gp path] (1.504,3.721)--(11.947,3.721);
\gpcolor{color=gp lt color border}
\gpsetlinetype{gp lt border}
\draw[gp path] (1.504,3.721)--(1.684,3.721);
\draw[gp path] (11.947,3.721)--(11.767,3.721);
\node[gp node right] at (1.320,3.721) { 100};
\gpcolor{color=gp lt color axes}
\gpsetlinetype{gp lt axes}
\draw[gp path] (1.504,5.089)--(11.947,5.089);
\gpcolor{color=gp lt color border}
\gpsetlinetype{gp lt border}
\draw[gp path] (1.504,5.089)--(1.684,5.089);
\draw[gp path] (11.947,5.089)--(11.767,5.089);
\node[gp node right] at (1.320,5.089) { 150};
\gpcolor{color=gp lt color axes}
\gpsetlinetype{gp lt axes}
\draw[gp path] (1.504,6.457)--(11.947,6.457);
\gpcolor{color=gp lt color border}
\gpsetlinetype{gp lt border}
\draw[gp path] (1.504,6.457)--(1.684,6.457);
\draw[gp path] (11.947,6.457)--(11.767,6.457);
\node[gp node right] at (1.320,6.457) { 200};
\gpcolor{color=gp lt color axes}
\gpsetlinetype{gp lt axes}
\draw[gp path] (1.504,7.825)--(11.947,7.825);
\gpcolor{color=gp lt color border}
\gpsetlinetype{gp lt border}
\draw[gp path] (1.504,7.825)--(1.684,7.825);
\draw[gp path] (11.947,7.825)--(11.767,7.825);
\node[gp node right] at (1.320,7.825) { 250};
\gpcolor{color=gp lt color axes}
\gpsetlinetype{gp lt axes}
\draw[gp path] (1.504,0.985)--(1.504,7.825);
\gpcolor{color=gp lt color border}
\gpsetlinetype{gp lt border}
\draw[gp path] (1.504,0.985)--(1.504,1.165);
\draw[gp path] (1.504,7.825)--(1.504,7.645);
\node[gp node center] at (1.504,0.677) { 0};
\gpcolor{color=gp lt color axes}
\gpsetlinetype{gp lt axes}
\draw[gp path] (2.664,0.985)--(2.664,7.825);
\gpcolor{color=gp lt color border}
\gpsetlinetype{gp lt border}
\draw[gp path] (2.664,0.985)--(2.664,1.165);
\draw[gp path] (2.664,7.825)--(2.664,7.645);
\node[gp node center] at (2.664,0.677) { 5};
\gpcolor{color=gp lt color axes}
\gpsetlinetype{gp lt axes}
\draw[gp path] (3.825,0.985)--(3.825,7.825);
\gpcolor{color=gp lt color border}
\gpsetlinetype{gp lt border}
\draw[gp path] (3.825,0.985)--(3.825,1.165);
\draw[gp path] (3.825,7.825)--(3.825,7.645);
\node[gp node center] at (3.825,0.677) { 10};
\gpcolor{color=gp lt color axes}
\gpsetlinetype{gp lt axes}
\draw[gp path] (4.985,0.985)--(4.985,7.825);
\gpcolor{color=gp lt color border}
\gpsetlinetype{gp lt border}
\draw[gp path] (4.985,0.985)--(4.985,1.165);
\draw[gp path] (4.985,7.825)--(4.985,7.645);
\node[gp node center] at (4.985,0.677) { 15};
\gpcolor{color=gp lt color axes}
\gpsetlinetype{gp lt axes}
\draw[gp path] (6.145,0.985)--(6.145,7.825);
\gpcolor{color=gp lt color border}
\gpsetlinetype{gp lt border}
\draw[gp path] (6.145,0.985)--(6.145,1.165);
\draw[gp path] (6.145,7.825)--(6.145,7.645);
\node[gp node center] at (6.145,0.677) { 20};
\gpcolor{color=gp lt color axes}
\gpsetlinetype{gp lt axes}
\draw[gp path] (7.306,0.985)--(7.306,7.825);
\gpcolor{color=gp lt color border}
\gpsetlinetype{gp lt border}
\draw[gp path] (7.306,0.985)--(7.306,1.165);
\draw[gp path] (7.306,7.825)--(7.306,7.645);
\node[gp node center] at (7.306,0.677) { 25};
\gpcolor{color=gp lt color axes}
\gpsetlinetype{gp lt axes}
\draw[gp path] (8.466,0.985)--(8.466,1.165);
\draw[gp path] (8.466,2.397)--(8.466,7.825);
\gpcolor{color=gp lt color border}
\gpsetlinetype{gp lt border}
\draw[gp path] (8.466,0.985)--(8.466,1.165);
\draw[gp path] (8.466,7.825)--(8.466,7.645);
\node[gp node center] at (8.466,0.677) { 30};
\gpcolor{color=gp lt color axes}
\gpsetlinetype{gp lt axes}
\draw[gp path] (9.626,0.985)--(9.626,1.165);
\draw[gp path] (9.626,2.397)--(9.626,7.825);
\gpcolor{color=gp lt color border}
\gpsetlinetype{gp lt border}
\draw[gp path] (9.626,0.985)--(9.626,1.165);
\draw[gp path] (9.626,7.825)--(9.626,7.645);
\node[gp node center] at (9.626,0.677) { 35};
\gpcolor{color=gp lt color axes}
\gpsetlinetype{gp lt axes}
\draw[gp path] (10.787,0.985)--(10.787,1.165);
\draw[gp path] (10.787,2.397)--(10.787,7.825);
\gpcolor{color=gp lt color border}
\gpsetlinetype{gp lt border}
\draw[gp path] (10.787,0.985)--(10.787,1.165);
\draw[gp path] (10.787,7.825)--(10.787,7.645);
\node[gp node center] at (10.787,0.677) { 40};
\gpcolor{color=gp lt color axes}
\gpsetlinetype{gp lt axes}
\draw[gp path] (11.947,0.985)--(11.947,7.825);
\gpcolor{color=gp lt color border}
\gpsetlinetype{gp lt border}
\draw[gp path] (11.947,0.985)--(11.947,1.165);
\draw[gp path] (11.947,7.825)--(11.947,7.645);
\node[gp node center] at (11.947,0.677) { 45};
\draw[gp path] (1.504,7.825)--(1.504,0.985)--(11.947,0.985)--(11.947,7.825)--cycle;
\node[gp node center,rotate=-270] at (0.246,4.405) {Allungamento ($10^{-5}	m$)};
\node[gp node center] at (6.725,0.215) {Forza (N)};
\node[gp node center] at (6.725,8.287) {Distribuzione dei dati: E5};
\node[gp node left] at (8.271,2.243) {Dati andata};
\gpcolor{color=gp lt color 0}
\gpsetlinetype{gp lt plot 0}
\draw[gp path] (10.663,2.243)--(11.579,2.243);
\draw[gp path] (10.663,2.333)--(10.663,2.153);
\draw[gp path] (11.579,2.333)--(11.579,2.153);
\draw[gp path] (2.319,0.985)--(2.499,0.985);
\draw[gp path] (2.319,0.985)--(2.499,0.985);
\draw[gp path] (3.224,1.601)--(3.404,1.601);
\draw[gp path] (3.224,1.601)--(3.404,1.601);
\draw[gp path] (4.152,2.189)--(4.332,2.189);
\draw[gp path] (4.152,2.189)--(4.332,2.189);
\draw[gp path] (5.057,2.763)--(5.237,2.763);
\draw[gp path] (5.057,2.763)--(5.237,2.763);
\draw[gp path] (5.963,3.338)--(6.143,3.338);
\draw[gp path] (5.963,3.338)--(6.143,3.338);
\draw[gp path] (6.868,3.926)--(7.048,3.926);
\draw[gp path] (6.868,3.926)--(7.048,3.926);
\draw[gp path] (7.773,4.501)--(7.953,4.501);
\draw[gp path] (7.773,4.501)--(7.953,4.501);
\draw[gp path] (8.701,5.062)--(8.881,5.062);
\draw[gp path] (8.701,5.062)--(8.881,5.062);
\draw[gp path] (9.606,5.636)--(9.786,5.636);
\draw[gp path] (9.606,5.636)--(9.786,5.636);
\draw[gp path] (10.511,6.238)--(10.691,6.238);
\draw[gp path] (10.511,6.238)--(10.691,6.238);
\draw[gp path] (11.416,6.785)--(11.596,6.785);
\draw[gp path] (11.416,6.785)--(11.596,6.785);
\gpsetpointsize{4.00}
\gppoint{gp mark 1}{(2.409,0.985)}
\gppoint{gp mark 1}{(3.314,1.601)}
\gppoint{gp mark 1}{(4.242,2.189)}
\gppoint{gp mark 1}{(5.147,2.763)}
\gppoint{gp mark 1}{(6.053,3.338)}
\gppoint{gp mark 1}{(6.958,3.926)}
\gppoint{gp mark 1}{(7.863,4.501)}
\gppoint{gp mark 1}{(8.791,5.062)}
\gppoint{gp mark 1}{(9.696,5.636)}
\gppoint{gp mark 1}{(10.601,6.238)}
\gppoint{gp mark 1}{(11.506,6.785)}
\gppoint{gp mark 1}{(11.121,2.243)}
\gpcolor{color=gp lt color border}
\node[gp node left] at (8.271,1.935) {Dati ritorno};
\gpcolor{color=gp lt color 1}
\gpsetlinetype{gp lt plot 1}
\draw[gp path] (10.663,1.935)--(11.579,1.935);
\draw[gp path] (10.663,2.025)--(10.663,1.845);
\draw[gp path] (11.579,2.025)--(11.579,1.845);
\draw[gp path] (11.416,6.785)--(11.596,6.785);
\draw[gp path] (11.416,6.785)--(11.596,6.785);
\draw[gp path] (10.511,6.279)--(10.691,6.279);
\draw[gp path] (10.511,6.279)--(10.691,6.279);
\draw[gp path] (9.606,5.705)--(9.786,5.705);
\draw[gp path] (9.606,5.705)--(9.786,5.705);
\draw[gp path] (8.701,5.116)--(8.881,5.116);
\draw[gp path] (8.701,5.116)--(8.881,5.116);
\draw[gp path] (7.773,4.569)--(7.953,4.569);
\draw[gp path] (7.773,4.569)--(7.953,4.569);
\draw[gp path] (6.868,3.995)--(7.048,3.995);
\draw[gp path] (6.868,3.995)--(7.048,3.995);
\draw[gp path] (5.963,3.420)--(6.143,3.420);
\draw[gp path] (5.963,3.420)--(6.143,3.420);
\draw[gp path] (5.057,2.818)--(5.237,2.818);
\draw[gp path] (5.057,2.818)--(5.237,2.818);
\draw[gp path] (4.152,2.517)--(4.332,2.517);
\draw[gp path] (4.152,2.517)--(4.332,2.517);
\draw[gp path] (3.224,1.642)--(3.404,1.642);
\draw[gp path] (3.224,1.642)--(3.404,1.642);
\draw[gp path] (2.319,1.012)--(2.499,1.012);
\draw[gp path] (2.319,1.012)--(2.499,1.012);
\gppoint{gp mark 2}{(11.506,6.785)}
\gppoint{gp mark 2}{(10.601,6.279)}
\gppoint{gp mark 2}{(9.696,5.705)}
\gppoint{gp mark 2}{(8.791,5.116)}
\gppoint{gp mark 2}{(7.863,4.569)}
\gppoint{gp mark 2}{(6.958,3.995)}
\gppoint{gp mark 2}{(6.053,3.420)}
\gppoint{gp mark 2}{(5.147,2.818)}
\gppoint{gp mark 2}{(4.242,2.517)}
\gppoint{gp mark 2}{(3.314,1.642)}
\gppoint{gp mark 2}{(2.409,1.012)}
\gppoint{gp mark 2}{(11.121,1.935)}
\gpcolor{color=gp lt color border}
\node[gp node left] at (8.271,1.627) {Andata};
\gpcolor{color=gp lt color 2}
\gpsetlinetype{gp lt plot 2}
\draw[gp path] (10.663,1.627)--(11.579,1.627);
\draw[gp path] (2.409,1.016)--(2.501,1.074)--(2.593,1.133)--(2.685,1.191)--(2.777,1.250)%
  --(2.869,1.308)--(2.960,1.367)--(3.052,1.425)--(3.144,1.483)--(3.236,1.542)--(3.328,1.600)%
  --(3.420,1.659)--(3.512,1.717)--(3.604,1.776)--(3.696,1.834)--(3.787,1.893)--(3.879,1.951)%
  --(3.971,2.009)--(4.063,2.068)--(4.155,2.126)--(4.247,2.185)--(4.339,2.243)--(4.431,2.302)%
  --(4.523,2.360)--(4.614,2.418)--(4.706,2.477)--(4.798,2.535)--(4.890,2.594)--(4.982,2.652)%
  --(5.074,2.711)--(5.166,2.769)--(5.258,2.828)--(5.350,2.886)--(5.441,2.944)--(5.533,3.003)%
  --(5.625,3.061)--(5.717,3.120)--(5.809,3.178)--(5.901,3.237)--(5.993,3.295)--(6.085,3.353)%
  --(6.177,3.412)--(6.268,3.470)--(6.360,3.529)--(6.452,3.587)--(6.544,3.646)--(6.636,3.704)%
  --(6.728,3.763)--(6.820,3.821)--(6.912,3.879)--(7.004,3.938)--(7.095,3.996)--(7.187,4.055)%
  --(7.279,4.113)--(7.371,4.172)--(7.463,4.230)--(7.555,4.288)--(7.647,4.347)--(7.739,4.405)%
  --(7.831,4.464)--(7.922,4.522)--(8.014,4.581)--(8.106,4.639)--(8.198,4.697)--(8.290,4.756)%
  --(8.382,4.814)--(8.474,4.873)--(8.566,4.931)--(8.658,4.990)--(8.749,5.048)--(8.841,5.107)%
  --(8.933,5.165)--(9.025,5.223)--(9.117,5.282)--(9.209,5.340)--(9.301,5.399)--(9.393,5.457)%
  --(9.485,5.516)--(9.576,5.574)--(9.668,5.632)--(9.760,5.691)--(9.852,5.749)--(9.944,5.808)%
  --(10.036,5.866)--(10.128,5.925)--(10.220,5.983)--(10.312,6.042)--(10.403,6.100)--(10.495,6.158)%
  --(10.587,6.217)--(10.679,6.275)--(10.771,6.334)--(10.863,6.392)--(10.955,6.451)--(11.047,6.509)%
  --(11.139,6.567)--(11.230,6.626)--(11.322,6.684)--(11.414,6.743)--(11.506,6.801);
\gpcolor{color=gp lt color border}
\node[gp node left] at (8.271,1.319) {Ritorno};
\gpcolor{color=gp lt color 3}
\gpsetlinetype{gp lt plot 3}
\draw[gp path] (10.663,1.319)--(11.579,1.319);
\draw[gp path] (2.409,1.134)--(2.501,1.192)--(2.593,1.249)--(2.685,1.307)--(2.777,1.364)%
  --(2.869,1.422)--(2.960,1.479)--(3.052,1.537)--(3.144,1.595)--(3.236,1.652)--(3.328,1.710)%
  --(3.420,1.767)--(3.512,1.825)--(3.604,1.883)--(3.696,1.940)--(3.787,1.998)--(3.879,2.055)%
  --(3.971,2.113)--(4.063,2.171)--(4.155,2.228)--(4.247,2.286)--(4.339,2.343)--(4.431,2.401)%
  --(4.523,2.458)--(4.614,2.516)--(4.706,2.574)--(4.798,2.631)--(4.890,2.689)--(4.982,2.746)%
  --(5.074,2.804)--(5.166,2.862)--(5.258,2.919)--(5.350,2.977)--(5.441,3.034)--(5.533,3.092)%
  --(5.625,3.149)--(5.717,3.207)--(5.809,3.265)--(5.901,3.322)--(5.993,3.380)--(6.085,3.437)%
  --(6.177,3.495)--(6.268,3.553)--(6.360,3.610)--(6.452,3.668)--(6.544,3.725)--(6.636,3.783)%
  --(6.728,3.841)--(6.820,3.898)--(6.912,3.956)--(7.004,4.013)--(7.095,4.071)--(7.187,4.128)%
  --(7.279,4.186)--(7.371,4.244)--(7.463,4.301)--(7.555,4.359)--(7.647,4.416)--(7.739,4.474)%
  --(7.831,4.532)--(7.922,4.589)--(8.014,4.647)--(8.106,4.704)--(8.198,4.762)--(8.290,4.819)%
  --(8.382,4.877)--(8.474,4.935)--(8.566,4.992)--(8.658,5.050)--(8.749,5.107)--(8.841,5.165)%
  --(8.933,5.223)--(9.025,5.280)--(9.117,5.338)--(9.209,5.395)--(9.301,5.453)--(9.393,5.511)%
  --(9.485,5.568)--(9.576,5.626)--(9.668,5.683)--(9.760,5.741)--(9.852,5.798)--(9.944,5.856)%
  --(10.036,5.914)--(10.128,5.971)--(10.220,6.029)--(10.312,6.086)--(10.403,6.144)--(10.495,6.202)%
  --(10.587,6.259)--(10.679,6.317)--(10.771,6.374)--(10.863,6.432)--(10.955,6.490)--(11.047,6.547)%
  --(11.139,6.605)--(11.230,6.662)--(11.322,6.720)--(11.414,6.777)--(11.506,6.835);
\gpcolor{color=gp lt color border}
\gpsetlinetype{gp lt border}
\draw[gp path] (1.504,7.825)--(1.504,0.985)--(11.947,0.985)--(11.947,7.825)--cycle;
%% coordinates of the plot area
\gpdefrectangularnode{gp plot 1}{\pgfpoint{1.504cm}{0.985cm}}{\pgfpoint{11.947cm}{7.825cm}}
\end{tikzpicture}
%% gnuplot variables

\caption{Spazio!}
\label{fig:spazio1}
\end{figure}

\begin{figure}[H]
    \centering
\begin{tikzpicture}[gnuplot]
%% generated with GNUPLOT 4.6p0 (Lua 5.1; terminal rev. 99, script rev. 100)
%% Tue 25 Mar 2014 05:11:20 PM CET
\path (0.000,0.000) rectangle (12.500,8.750);
\gpcolor{color=gp lt color axes}
\gpsetlinetype{gp lt axes}
\gpsetlinewidth{1.00}
\draw[gp path] (1.504,0.985)--(11.947,0.985);
\gpcolor{color=gp lt color border}
\gpsetlinetype{gp lt border}
\draw[gp path] (1.504,0.985)--(1.684,0.985);
\draw[gp path] (11.947,0.985)--(11.767,0.985);
\node[gp node right] at (1.320,0.985) { 0};
\gpcolor{color=gp lt color axes}
\gpsetlinetype{gp lt axes}
\draw[gp path] (1.504,1.745)--(8.271,1.745);
\draw[gp path] (11.763,1.745)--(11.947,1.745);
\gpcolor{color=gp lt color border}
\gpsetlinetype{gp lt border}
\draw[gp path] (1.504,1.745)--(1.684,1.745);
\draw[gp path] (11.947,1.745)--(11.767,1.745);
\node[gp node right] at (1.320,1.745) { 20};
\gpcolor{color=gp lt color axes}
\gpsetlinetype{gp lt axes}
\draw[gp path] (1.504,2.505)--(11.947,2.505);
\gpcolor{color=gp lt color border}
\gpsetlinetype{gp lt border}
\draw[gp path] (1.504,2.505)--(1.684,2.505);
\draw[gp path] (11.947,2.505)--(11.767,2.505);
\node[gp node right] at (1.320,2.505) { 40};
\gpcolor{color=gp lt color axes}
\gpsetlinetype{gp lt axes}
\draw[gp path] (1.504,3.265)--(11.947,3.265);
\gpcolor{color=gp lt color border}
\gpsetlinetype{gp lt border}
\draw[gp path] (1.504,3.265)--(1.684,3.265);
\draw[gp path] (11.947,3.265)--(11.767,3.265);
\node[gp node right] at (1.320,3.265) { 60};
\gpcolor{color=gp lt color axes}
\gpsetlinetype{gp lt axes}
\draw[gp path] (1.504,4.025)--(11.947,4.025);
\gpcolor{color=gp lt color border}
\gpsetlinetype{gp lt border}
\draw[gp path] (1.504,4.025)--(1.684,4.025);
\draw[gp path] (11.947,4.025)--(11.767,4.025);
\node[gp node right] at (1.320,4.025) { 80};
\gpcolor{color=gp lt color axes}
\gpsetlinetype{gp lt axes}
\draw[gp path] (1.504,4.785)--(11.947,4.785);
\gpcolor{color=gp lt color border}
\gpsetlinetype{gp lt border}
\draw[gp path] (1.504,4.785)--(1.684,4.785);
\draw[gp path] (11.947,4.785)--(11.767,4.785);
\node[gp node right] at (1.320,4.785) { 100};
\gpcolor{color=gp lt color axes}
\gpsetlinetype{gp lt axes}
\draw[gp path] (1.504,5.545)--(11.947,5.545);
\gpcolor{color=gp lt color border}
\gpsetlinetype{gp lt border}
\draw[gp path] (1.504,5.545)--(1.684,5.545);
\draw[gp path] (11.947,5.545)--(11.767,5.545);
\node[gp node right] at (1.320,5.545) { 120};
\gpcolor{color=gp lt color axes}
\gpsetlinetype{gp lt axes}
\draw[gp path] (1.504,6.305)--(11.947,6.305);
\gpcolor{color=gp lt color border}
\gpsetlinetype{gp lt border}
\draw[gp path] (1.504,6.305)--(1.684,6.305);
\draw[gp path] (11.947,6.305)--(11.767,6.305);
\node[gp node right] at (1.320,6.305) { 140};
\gpcolor{color=gp lt color axes}
\gpsetlinetype{gp lt axes}
\draw[gp path] (1.504,7.065)--(11.947,7.065);
\gpcolor{color=gp lt color border}
\gpsetlinetype{gp lt border}
\draw[gp path] (1.504,7.065)--(1.684,7.065);
\draw[gp path] (11.947,7.065)--(11.767,7.065);
\node[gp node right] at (1.320,7.065) { 160};
\gpcolor{color=gp lt color axes}
\gpsetlinetype{gp lt axes}
\draw[gp path] (1.504,7.825)--(11.947,7.825);
\gpcolor{color=gp lt color border}
\gpsetlinetype{gp lt border}
\draw[gp path] (1.504,7.825)--(1.684,7.825);
\draw[gp path] (11.947,7.825)--(11.767,7.825);
\node[gp node right] at (1.320,7.825) { 180};
\gpcolor{color=gp lt color axes}
\gpsetlinetype{gp lt axes}
\draw[gp path] (1.504,0.985)--(1.504,7.825);
\gpcolor{color=gp lt color border}
\gpsetlinetype{gp lt border}
\draw[gp path] (1.504,0.985)--(1.504,1.165);
\draw[gp path] (1.504,7.825)--(1.504,7.645);
\node[gp node center] at (1.504,0.677) { 0};
\gpcolor{color=gp lt color axes}
\gpsetlinetype{gp lt axes}
\draw[gp path] (2.664,0.985)--(2.664,7.825);
\gpcolor{color=gp lt color border}
\gpsetlinetype{gp lt border}
\draw[gp path] (2.664,0.985)--(2.664,1.165);
\draw[gp path] (2.664,7.825)--(2.664,7.645);
\node[gp node center] at (2.664,0.677) { 5};
\gpcolor{color=gp lt color axes}
\gpsetlinetype{gp lt axes}
\draw[gp path] (3.825,0.985)--(3.825,7.825);
\gpcolor{color=gp lt color border}
\gpsetlinetype{gp lt border}
\draw[gp path] (3.825,0.985)--(3.825,1.165);
\draw[gp path] (3.825,7.825)--(3.825,7.645);
\node[gp node center] at (3.825,0.677) { 10};
\gpcolor{color=gp lt color axes}
\gpsetlinetype{gp lt axes}
\draw[gp path] (4.985,0.985)--(4.985,7.825);
\gpcolor{color=gp lt color border}
\gpsetlinetype{gp lt border}
\draw[gp path] (4.985,0.985)--(4.985,1.165);
\draw[gp path] (4.985,7.825)--(4.985,7.645);
\node[gp node center] at (4.985,0.677) { 15};
\gpcolor{color=gp lt color axes}
\gpsetlinetype{gp lt axes}
\draw[gp path] (6.145,0.985)--(6.145,7.825);
\gpcolor{color=gp lt color border}
\gpsetlinetype{gp lt border}
\draw[gp path] (6.145,0.985)--(6.145,1.165);
\draw[gp path] (6.145,7.825)--(6.145,7.645);
\node[gp node center] at (6.145,0.677) { 20};
\gpcolor{color=gp lt color axes}
\gpsetlinetype{gp lt axes}
\draw[gp path] (7.306,0.985)--(7.306,7.825);
\gpcolor{color=gp lt color border}
\gpsetlinetype{gp lt border}
\draw[gp path] (7.306,0.985)--(7.306,1.165);
\draw[gp path] (7.306,7.825)--(7.306,7.645);
\node[gp node center] at (7.306,0.677) { 25};
\gpcolor{color=gp lt color axes}
\gpsetlinetype{gp lt axes}
\draw[gp path] (8.466,0.985)--(8.466,1.165);
\draw[gp path] (8.466,2.397)--(8.466,7.825);
\gpcolor{color=gp lt color border}
\gpsetlinetype{gp lt border}
\draw[gp path] (8.466,0.985)--(8.466,1.165);
\draw[gp path] (8.466,7.825)--(8.466,7.645);
\node[gp node center] at (8.466,0.677) { 30};
\gpcolor{color=gp lt color axes}
\gpsetlinetype{gp lt axes}
\draw[gp path] (9.626,0.985)--(9.626,1.165);
\draw[gp path] (9.626,2.397)--(9.626,7.825);
\gpcolor{color=gp lt color border}
\gpsetlinetype{gp lt border}
\draw[gp path] (9.626,0.985)--(9.626,1.165);
\draw[gp path] (9.626,7.825)--(9.626,7.645);
\node[gp node center] at (9.626,0.677) { 35};
\gpcolor{color=gp lt color axes}
\gpsetlinetype{gp lt axes}
\draw[gp path] (10.787,0.985)--(10.787,1.165);
\draw[gp path] (10.787,2.397)--(10.787,7.825);
\gpcolor{color=gp lt color border}
\gpsetlinetype{gp lt border}
\draw[gp path] (10.787,0.985)--(10.787,1.165);
\draw[gp path] (10.787,7.825)--(10.787,7.645);
\node[gp node center] at (10.787,0.677) { 40};
\gpcolor{color=gp lt color axes}
\gpsetlinetype{gp lt axes}
\draw[gp path] (11.947,0.985)--(11.947,7.825);
\gpcolor{color=gp lt color border}
\gpsetlinetype{gp lt border}
\draw[gp path] (11.947,0.985)--(11.947,1.165);
\draw[gp path] (11.947,7.825)--(11.947,7.645);
\node[gp node center] at (11.947,0.677) { 45};
\draw[gp path] (1.504,7.825)--(1.504,0.985)--(11.947,0.985)--(11.947,7.825)--cycle;
\node[gp node center,rotate=-270] at (0.246,4.405) {Allungamento ($10^{-5} [m]$)};
\node[gp node center] at (6.725,0.215) {Forza (N)};
\node[gp node center] at (6.725,8.287) {Distribuzione dei dati: E8};
\node[gp node left] at (8.271,2.243) {Dati andata};
\gpcolor{color=gp lt color 0}
\gpsetlinetype{gp lt plot 0}
\draw[gp path] (10.663,2.243)--(11.579,2.243);
\draw[gp path] (10.663,2.333)--(10.663,2.153);
\draw[gp path] (11.579,2.333)--(11.579,2.153);
\draw[gp path] (2.319,0.985)--(2.499,0.985);
\draw[gp path] (2.319,0.985)--(2.499,0.985);
\draw[gp path] (3.224,1.669)--(3.404,1.669);
\draw[gp path] (3.224,1.669)--(3.404,1.669);
\draw[gp path] (4.152,2.353)--(4.332,2.353);
\draw[gp path] (4.152,2.353)--(4.332,2.353);
\draw[gp path] (5.057,3.056)--(5.237,3.056);
\draw[gp path] (5.057,3.056)--(5.237,3.056);
\draw[gp path] (5.963,3.683)--(6.143,3.683);
\draw[gp path] (5.963,3.683)--(6.143,3.683);
\draw[gp path] (6.868,4.367)--(7.048,4.367);
\draw[gp path] (6.868,4.367)--(7.048,4.367);
\draw[gp path] (7.773,4.994)--(7.953,4.994);
\draw[gp path] (7.773,4.994)--(7.953,4.994);
\draw[gp path] (8.701,5.659)--(8.881,5.659);
\draw[gp path] (8.701,5.659)--(8.881,5.659);
\draw[gp path] (9.606,6.381)--(9.786,6.381);
\draw[gp path] (9.606,6.381)--(9.786,6.381);
\draw[gp path] (10.511,7.008)--(10.691,7.008);
\draw[gp path] (10.511,7.008)--(10.691,7.008);
\draw[gp path] (11.416,7.654)--(11.596,7.654);
\draw[gp path] (11.416,7.654)--(11.596,7.654);
\gpsetpointsize{4.00}
\gppoint{gp mark 1}{(2.409,0.985)}
\gppoint{gp mark 1}{(3.314,1.669)}
\gppoint{gp mark 1}{(4.242,2.353)}
\gppoint{gp mark 1}{(5.147,3.056)}
\gppoint{gp mark 1}{(6.053,3.683)}
\gppoint{gp mark 1}{(6.958,4.367)}
\gppoint{gp mark 1}{(7.863,4.994)}
\gppoint{gp mark 1}{(8.791,5.659)}
\gppoint{gp mark 1}{(9.696,6.381)}
\gppoint{gp mark 1}{(10.601,7.008)}
\gppoint{gp mark 1}{(11.506,7.654)}
\gppoint{gp mark 1}{(11.121,2.243)}
\gpcolor{color=gp lt color border}
\node[gp node left] at (8.271,1.935) {Dati ritorno};
\gpcolor{color=gp lt color 1}
\gpsetlinetype{gp lt plot 1}
\draw[gp path] (10.663,1.935)--(11.579,1.935);
\draw[gp path] (10.663,2.025)--(10.663,1.845);
\draw[gp path] (11.579,2.025)--(11.579,1.845);
\draw[gp path] (11.416,7.673)--(11.596,7.673);
\draw[gp path] (11.416,7.673)--(11.596,7.673);
\draw[gp path] (10.511,7.027)--(10.691,7.027);
\draw[gp path] (10.511,7.027)--(10.691,7.027);
\draw[gp path] (9.606,6.362)--(9.786,6.362);
\draw[gp path] (9.606,6.362)--(9.786,6.362);
\draw[gp path] (8.701,5.697)--(8.881,5.697);
\draw[gp path] (8.701,5.697)--(8.881,5.697);
\draw[gp path] (7.773,5.013)--(7.953,5.013);
\draw[gp path] (7.773,5.013)--(7.953,5.013);
\draw[gp path] (6.868,4.234)--(7.048,4.234);
\draw[gp path] (6.868,4.234)--(7.048,4.234);
\draw[gp path] (5.963,3.683)--(6.143,3.683);
\draw[gp path] (5.963,3.683)--(6.143,3.683);
\draw[gp path] (5.057,3.037)--(5.237,3.037);
\draw[gp path] (5.057,3.037)--(5.237,3.037);
\draw[gp path] (4.152,2.334)--(4.332,2.334);
\draw[gp path] (4.152,2.334)--(4.332,2.334);
\draw[gp path] (3.224,1.669)--(3.404,1.669);
\draw[gp path] (3.224,1.669)--(3.404,1.669);
\draw[gp path] (2.319,0.985)--(2.499,0.985);
\draw[gp path] (2.319,0.985)--(2.499,0.985);
\gppoint{gp mark 2}{(11.506,7.673)}
\gppoint{gp mark 2}{(10.601,7.027)}
\gppoint{gp mark 2}{(9.696,6.362)}
\gppoint{gp mark 2}{(8.791,5.697)}
\gppoint{gp mark 2}{(7.863,5.013)}
\gppoint{gp mark 2}{(6.958,4.234)}
\gppoint{gp mark 2}{(6.053,3.683)}
\gppoint{gp mark 2}{(5.147,3.037)}
\gppoint{gp mark 2}{(4.242,2.334)}
\gppoint{gp mark 2}{(3.314,1.669)}
\gppoint{gp mark 2}{(2.409,0.985)}
\gppoint{gp mark 2}{(11.121,1.935)}
\gpcolor{color=gp lt color border}
\node[gp node left] at (8.271,1.627) {Andata};
\gpcolor{color=gp lt color 2}
\gpsetlinetype{gp lt plot 2}
\draw[gp path] (10.663,1.627)--(11.579,1.627);
\draw[gp path] (2.409,1.012)--(2.501,1.079)--(2.593,1.147)--(2.685,1.214)--(2.777,1.281)%
  --(2.869,1.349)--(2.960,1.416)--(3.052,1.483)--(3.144,1.550)--(3.236,1.618)--(3.328,1.685)%
  --(3.420,1.752)--(3.512,1.820)--(3.604,1.887)--(3.696,1.954)--(3.787,2.021)--(3.879,2.089)%
  --(3.971,2.156)--(4.063,2.223)--(4.155,2.291)--(4.247,2.358)--(4.339,2.425)--(4.431,2.493)%
  --(4.523,2.560)--(4.614,2.627)--(4.706,2.694)--(4.798,2.762)--(4.890,2.829)--(4.982,2.896)%
  --(5.074,2.964)--(5.166,3.031)--(5.258,3.098)--(5.350,3.166)--(5.441,3.233)--(5.533,3.300)%
  --(5.625,3.367)--(5.717,3.435)--(5.809,3.502)--(5.901,3.569)--(5.993,3.637)--(6.085,3.704)%
  --(6.177,3.771)--(6.268,3.838)--(6.360,3.906)--(6.452,3.973)--(6.544,4.040)--(6.636,4.108)%
  --(6.728,4.175)--(6.820,4.242)--(6.912,4.310)--(7.004,4.377)--(7.095,4.444)--(7.187,4.511)%
  --(7.279,4.579)--(7.371,4.646)--(7.463,4.713)--(7.555,4.781)--(7.647,4.848)--(7.739,4.915)%
  --(7.831,4.982)--(7.922,5.050)--(8.014,5.117)--(8.106,5.184)--(8.198,5.252)--(8.290,5.319)%
  --(8.382,5.386)--(8.474,5.454)--(8.566,5.521)--(8.658,5.588)--(8.749,5.655)--(8.841,5.723)%
  --(8.933,5.790)--(9.025,5.857)--(9.117,5.925)--(9.209,5.992)--(9.301,6.059)--(9.393,6.127)%
  --(9.485,6.194)--(9.576,6.261)--(9.668,6.328)--(9.760,6.396)--(9.852,6.463)--(9.944,6.530)%
  --(10.036,6.598)--(10.128,6.665)--(10.220,6.732)--(10.312,6.799)--(10.403,6.867)--(10.495,6.934)%
  --(10.587,7.001)--(10.679,7.069)--(10.771,7.136)--(10.863,7.203)--(10.955,7.271)--(11.047,7.338)%
  --(11.139,7.405)--(11.230,7.472)--(11.322,7.540)--(11.414,7.607)--(11.506,7.674);
\gpcolor{color=gp lt color border}
\node[gp node left] at (8.271,1.319) {Ritorno};
\gpcolor{color=gp lt color 3}
\gpsetlinetype{gp lt plot 3}
\draw[gp path] (10.663,1.319)--(11.579,1.319);
\draw[gp path] (2.409,0.990)--(2.501,1.057)--(2.593,1.125)--(2.685,1.192)--(2.777,1.260)%
  --(2.869,1.327)--(2.960,1.395)--(3.052,1.463)--(3.144,1.530)--(3.236,1.598)--(3.328,1.665)%
  --(3.420,1.733)--(3.512,1.800)--(3.604,1.868)--(3.696,1.936)--(3.787,2.003)--(3.879,2.071)%
  --(3.971,2.138)--(4.063,2.206)--(4.155,2.273)--(4.247,2.341)--(4.339,2.409)--(4.431,2.476)%
  --(4.523,2.544)--(4.614,2.611)--(4.706,2.679)--(4.798,2.747)--(4.890,2.814)--(4.982,2.882)%
  --(5.074,2.949)--(5.166,3.017)--(5.258,3.084)--(5.350,3.152)--(5.441,3.220)--(5.533,3.287)%
  --(5.625,3.355)--(5.717,3.422)--(5.809,3.490)--(5.901,3.557)--(5.993,3.625)--(6.085,3.693)%
  --(6.177,3.760)--(6.268,3.828)--(6.360,3.895)--(6.452,3.963)--(6.544,4.030)--(6.636,4.098)%
  --(6.728,4.166)--(6.820,4.233)--(6.912,4.301)--(7.004,4.368)--(7.095,4.436)--(7.187,4.503)%
  --(7.279,4.571)--(7.371,4.639)--(7.463,4.706)--(7.555,4.774)--(7.647,4.841)--(7.739,4.909)%
  --(7.831,4.976)--(7.922,5.044)--(8.014,5.112)--(8.106,5.179)--(8.198,5.247)--(8.290,5.314)%
  --(8.382,5.382)--(8.474,5.450)--(8.566,5.517)--(8.658,5.585)--(8.749,5.652)--(8.841,5.720)%
  --(8.933,5.787)--(9.025,5.855)--(9.117,5.923)--(9.209,5.990)--(9.301,6.058)--(9.393,6.125)%
  --(9.485,6.193)--(9.576,6.260)--(9.668,6.328)--(9.760,6.396)--(9.852,6.463)--(9.944,6.531)%
  --(10.036,6.598)--(10.128,6.666)--(10.220,6.733)--(10.312,6.801)--(10.403,6.869)--(10.495,6.936)%
  --(10.587,7.004)--(10.679,7.071)--(10.771,7.139)--(10.863,7.206)--(10.955,7.274)--(11.047,7.342)%
  --(11.139,7.409)--(11.230,7.477)--(11.322,7.544)--(11.414,7.612)--(11.506,7.680);
\gpcolor{color=gp lt color border}
\gpsetlinetype{gp lt border}
\draw[gp path] (1.504,7.825)--(1.504,0.985)--(11.947,0.985)--(11.947,7.825)--cycle;
%% coordinates of the plot area
\gpdefrectangularnode{gp plot 1}{\pgfpoint{1.504cm}{0.985cm}}{\pgfpoint{11.947cm}{7.825cm}}
\end{tikzpicture}
%% gnuplot variables

\caption{Spazio!}
\label{fig:spazio1}
\end{figure}

\begin{figure}[H]
    \centering
\begin{tikzpicture}[gnuplot]
%% generated with GNUPLOT 4.6p0 (Lua 5.1; terminal rev. 99, script rev. 100)
%% Tue 25 Mar 2014 10:00:25 AM CET
\path (0.000,0.000) rectangle (12.500,8.750);
\gpcolor{color=gp lt color axes}
\gpsetlinetype{gp lt axes}
\gpsetlinewidth{1.00}
\draw[gp path] (1.504,0.985)--(11.947,0.985);
\gpcolor{color=gp lt color border}
\gpsetlinetype{gp lt border}
\draw[gp path] (1.504,0.985)--(1.684,0.985);
\draw[gp path] (11.947,0.985)--(11.767,0.985);
\node[gp node right] at (1.320,0.985) {-50};
\gpcolor{color=gp lt color axes}
\gpsetlinetype{gp lt axes}
\draw[gp path] (1.504,1.962)--(8.271,1.962);
\draw[gp path] (11.763,1.962)--(11.947,1.962);
\gpcolor{color=gp lt color border}
\gpsetlinetype{gp lt border}
\draw[gp path] (1.504,1.962)--(1.684,1.962);
\draw[gp path] (11.947,1.962)--(11.767,1.962);
\node[gp node right] at (1.320,1.962) { 0};
\gpcolor{color=gp lt color axes}
\gpsetlinetype{gp lt axes}
\draw[gp path] (1.504,2.939)--(11.947,2.939);
\gpcolor{color=gp lt color border}
\gpsetlinetype{gp lt border}
\draw[gp path] (1.504,2.939)--(1.684,2.939);
\draw[gp path] (11.947,2.939)--(11.767,2.939);
\node[gp node right] at (1.320,2.939) { 50};
\gpcolor{color=gp lt color axes}
\gpsetlinetype{gp lt axes}
\draw[gp path] (1.504,3.916)--(11.947,3.916);
\gpcolor{color=gp lt color border}
\gpsetlinetype{gp lt border}
\draw[gp path] (1.504,3.916)--(1.684,3.916);
\draw[gp path] (11.947,3.916)--(11.767,3.916);
\node[gp node right] at (1.320,3.916) { 100};
\gpcolor{color=gp lt color axes}
\gpsetlinetype{gp lt axes}
\draw[gp path] (1.504,4.894)--(11.947,4.894);
\gpcolor{color=gp lt color border}
\gpsetlinetype{gp lt border}
\draw[gp path] (1.504,4.894)--(1.684,4.894);
\draw[gp path] (11.947,4.894)--(11.767,4.894);
\node[gp node right] at (1.320,4.894) { 150};
\gpcolor{color=gp lt color axes}
\gpsetlinetype{gp lt axes}
\draw[gp path] (1.504,5.871)--(11.947,5.871);
\gpcolor{color=gp lt color border}
\gpsetlinetype{gp lt border}
\draw[gp path] (1.504,5.871)--(1.684,5.871);
\draw[gp path] (11.947,5.871)--(11.767,5.871);
\node[gp node right] at (1.320,5.871) { 200};
\gpcolor{color=gp lt color axes}
\gpsetlinetype{gp lt axes}
\draw[gp path] (1.504,6.848)--(11.947,6.848);
\gpcolor{color=gp lt color border}
\gpsetlinetype{gp lt border}
\draw[gp path] (1.504,6.848)--(1.684,6.848);
\draw[gp path] (11.947,6.848)--(11.767,6.848);
\node[gp node right] at (1.320,6.848) { 250};
\gpcolor{color=gp lt color axes}
\gpsetlinetype{gp lt axes}
\draw[gp path] (1.504,7.825)--(11.947,7.825);
\gpcolor{color=gp lt color border}
\gpsetlinetype{gp lt border}
\draw[gp path] (1.504,7.825)--(1.684,7.825);
\draw[gp path] (11.947,7.825)--(11.767,7.825);
\node[gp node right] at (1.320,7.825) { 300};
\gpcolor{color=gp lt color axes}
\gpsetlinetype{gp lt axes}
\draw[gp path] (1.504,0.985)--(1.504,7.825);
\gpcolor{color=gp lt color border}
\gpsetlinetype{gp lt border}
\draw[gp path] (1.504,0.985)--(1.504,1.165);
\draw[gp path] (1.504,7.825)--(1.504,7.645);
\node[gp node center] at (1.504,0.677) { 0};
\gpcolor{color=gp lt color axes}
\gpsetlinetype{gp lt axes}
\draw[gp path] (2.664,0.985)--(2.664,7.825);
\gpcolor{color=gp lt color border}
\gpsetlinetype{gp lt border}
\draw[gp path] (2.664,0.985)--(2.664,1.165);
\draw[gp path] (2.664,7.825)--(2.664,7.645);
\node[gp node center] at (2.664,0.677) { 5};
\gpcolor{color=gp lt color axes}
\gpsetlinetype{gp lt axes}
\draw[gp path] (3.825,0.985)--(3.825,7.825);
\gpcolor{color=gp lt color border}
\gpsetlinetype{gp lt border}
\draw[gp path] (3.825,0.985)--(3.825,1.165);
\draw[gp path] (3.825,7.825)--(3.825,7.645);
\node[gp node center] at (3.825,0.677) { 10};
\gpcolor{color=gp lt color axes}
\gpsetlinetype{gp lt axes}
\draw[gp path] (4.985,0.985)--(4.985,7.825);
\gpcolor{color=gp lt color border}
\gpsetlinetype{gp lt border}
\draw[gp path] (4.985,0.985)--(4.985,1.165);
\draw[gp path] (4.985,7.825)--(4.985,7.645);
\node[gp node center] at (4.985,0.677) { 15};
\gpcolor{color=gp lt color axes}
\gpsetlinetype{gp lt axes}
\draw[gp path] (6.145,0.985)--(6.145,7.825);
\gpcolor{color=gp lt color border}
\gpsetlinetype{gp lt border}
\draw[gp path] (6.145,0.985)--(6.145,1.165);
\draw[gp path] (6.145,7.825)--(6.145,7.645);
\node[gp node center] at (6.145,0.677) { 20};
\gpcolor{color=gp lt color axes}
\gpsetlinetype{gp lt axes}
\draw[gp path] (7.306,0.985)--(7.306,7.825);
\gpcolor{color=gp lt color border}
\gpsetlinetype{gp lt border}
\draw[gp path] (7.306,0.985)--(7.306,1.165);
\draw[gp path] (7.306,7.825)--(7.306,7.645);
\node[gp node center] at (7.306,0.677) { 25};
\gpcolor{color=gp lt color axes}
\gpsetlinetype{gp lt axes}
\draw[gp path] (8.466,0.985)--(8.466,1.165);
\draw[gp path] (8.466,2.397)--(8.466,7.825);
\gpcolor{color=gp lt color border}
\gpsetlinetype{gp lt border}
\draw[gp path] (8.466,0.985)--(8.466,1.165);
\draw[gp path] (8.466,7.825)--(8.466,7.645);
\node[gp node center] at (8.466,0.677) { 30};
\gpcolor{color=gp lt color axes}
\gpsetlinetype{gp lt axes}
\draw[gp path] (9.626,0.985)--(9.626,1.165);
\draw[gp path] (9.626,2.397)--(9.626,7.825);
\gpcolor{color=gp lt color border}
\gpsetlinetype{gp lt border}
\draw[gp path] (9.626,0.985)--(9.626,1.165);
\draw[gp path] (9.626,7.825)--(9.626,7.645);
\node[gp node center] at (9.626,0.677) { 35};
\gpcolor{color=gp lt color axes}
\gpsetlinetype{gp lt axes}
\draw[gp path] (10.787,0.985)--(10.787,1.165);
\draw[gp path] (10.787,2.397)--(10.787,7.825);
\gpcolor{color=gp lt color border}
\gpsetlinetype{gp lt border}
\draw[gp path] (10.787,0.985)--(10.787,1.165);
\draw[gp path] (10.787,7.825)--(10.787,7.645);
\node[gp node center] at (10.787,0.677) { 40};
\gpcolor{color=gp lt color axes}
\gpsetlinetype{gp lt axes}
\draw[gp path] (11.947,0.985)--(11.947,7.825);
\gpcolor{color=gp lt color border}
\gpsetlinetype{gp lt border}
\draw[gp path] (11.947,0.985)--(11.947,1.165);
\draw[gp path] (11.947,7.825)--(11.947,7.645);
\node[gp node center] at (11.947,0.677) { 45};
\draw[gp path] (1.504,7.825)--(1.504,0.985)--(11.947,0.985)--(11.947,7.825)--cycle;
\node[gp node center,rotate=-270] at (0.246,4.405) {Allungamento ($10^{-5}	m$)};
\node[gp node center] at (6.725,0.215) {Forza (N)};
\node[gp node center] at (6.725,8.287) {Distribuzione dei dati: E13};
\node[gp node left] at (8.271,2.243) {Dati andata};
\gpcolor{color=gp lt color 0}
\gpsetlinetype{gp lt plot 0}
\draw[gp path] (10.663,2.243)--(11.579,2.243);
\draw[gp path] (10.663,2.333)--(10.663,2.153);
\draw[gp path] (11.579,2.333)--(11.579,2.153);
\draw[gp path] (2.319,1.962)--(2.499,1.962);
\draw[gp path] (2.319,1.962)--(2.499,1.962);
\draw[gp path] (3.224,2.490)--(3.404,2.490);
\draw[gp path] (3.224,2.490)--(3.404,2.490);
\draw[gp path] (4.152,3.105)--(4.332,3.105);
\draw[gp path] (4.152,3.105)--(4.332,3.105);
\draw[gp path] (5.057,3.555)--(5.237,3.555);
\draw[gp path] (5.057,3.555)--(5.237,3.555);
\draw[gp path] (5.963,4.092)--(6.143,4.092);
\draw[gp path] (5.963,4.092)--(6.143,4.092);
\draw[gp path] (6.868,4.620)--(7.048,4.620);
\draw[gp path] (6.868,4.620)--(7.048,4.620);
\draw[gp path] (7.773,5.167)--(7.953,5.167);
\draw[gp path] (7.773,5.167)--(7.953,5.167);
\draw[gp path] (8.701,5.695)--(8.881,5.695);
\draw[gp path] (8.701,5.695)--(8.881,5.695);
\draw[gp path] (9.606,6.242)--(9.786,6.242);
\draw[gp path] (9.606,6.242)--(9.786,6.242);
\draw[gp path] (10.511,6.770)--(10.691,6.770);
\draw[gp path] (10.511,6.770)--(10.691,6.770);
\draw[gp path] (11.416,7.327)--(11.596,7.327);
\draw[gp path] (11.416,7.327)--(11.596,7.327);
\gpsetpointsize{4.00}
\gppoint{gp mark 1}{(2.409,1.962)}
\gppoint{gp mark 1}{(3.314,2.490)}
\gppoint{gp mark 1}{(4.242,3.105)}
\gppoint{gp mark 1}{(5.147,3.555)}
\gppoint{gp mark 1}{(6.053,4.092)}
\gppoint{gp mark 1}{(6.958,4.620)}
\gppoint{gp mark 1}{(7.863,5.167)}
\gppoint{gp mark 1}{(8.791,5.695)}
\gppoint{gp mark 1}{(9.696,6.242)}
\gppoint{gp mark 1}{(10.601,6.770)}
\gppoint{gp mark 1}{(11.506,7.327)}
\gppoint{gp mark 1}{(11.121,2.243)}
\gpcolor{color=gp lt color border}
\node[gp node left] at (8.271,1.935) {Dati ritorno};
\gpcolor{color=gp lt color 1}
\gpsetlinetype{gp lt plot 1}
\draw[gp path] (10.663,1.935)--(11.579,1.935);
\draw[gp path] (10.663,2.025)--(10.663,1.845);
\draw[gp path] (11.579,2.025)--(11.579,1.845);
\draw[gp path] (11.416,7.336)--(11.596,7.336);
\draw[gp path] (11.416,7.336)--(11.596,7.336);
\draw[gp path] (10.511,6.779)--(10.691,6.779);
\draw[gp path] (10.511,6.779)--(10.691,6.779);
\draw[gp path] (9.606,6.222)--(9.786,6.222);
\draw[gp path] (9.606,6.222)--(9.786,6.222);
\draw[gp path] (8.701,5.675)--(8.881,5.675);
\draw[gp path] (8.701,5.675)--(8.881,5.675);
\draw[gp path] (7.773,5.157)--(7.953,5.157);
\draw[gp path] (7.773,5.157)--(7.953,5.157);
\draw[gp path] (6.868,4.600)--(7.048,4.600);
\draw[gp path] (6.868,4.600)--(7.048,4.600);
\draw[gp path] (5.963,4.073)--(6.143,4.073);
\draw[gp path] (5.963,4.073)--(6.143,4.073);
\draw[gp path] (5.057,3.545)--(5.237,3.545);
\draw[gp path] (5.057,3.545)--(5.237,3.545);
\draw[gp path] (4.152,3.096)--(4.332,3.096);
\draw[gp path] (4.152,3.096)--(4.332,3.096);
\draw[gp path] (3.224,2.470)--(3.404,2.470);
\draw[gp path] (3.224,2.470)--(3.404,2.470);
\draw[gp path] (2.319,1.962)--(2.499,1.962);
\draw[gp path] (2.319,1.962)--(2.499,1.962);
\gppoint{gp mark 2}{(11.506,7.336)}
\gppoint{gp mark 2}{(10.601,6.779)}
\gppoint{gp mark 2}{(9.696,6.222)}
\gppoint{gp mark 2}{(8.791,5.675)}
\gppoint{gp mark 2}{(7.863,5.157)}
\gppoint{gp mark 2}{(6.958,4.600)}
\gppoint{gp mark 2}{(6.053,4.073)}
\gppoint{gp mark 2}{(5.147,3.545)}
\gppoint{gp mark 2}{(4.242,3.096)}
\gppoint{gp mark 2}{(3.314,2.470)}
\gppoint{gp mark 2}{(2.409,1.962)}
\gppoint{gp mark 2}{(11.121,1.935)}
\gpcolor{color=gp lt color border}
\node[gp node left] at (8.271,1.627) {Andata};
\gpcolor{color=gp lt color 2}
\gpsetlinetype{gp lt plot 2}
\draw[gp path] (10.663,1.627)--(11.579,1.627);
\draw[gp path] (2.409,1.968)--(2.501,2.022)--(2.593,2.076)--(2.685,2.130)--(2.777,2.184)%
  --(2.869,2.238)--(2.960,2.292)--(3.052,2.346)--(3.144,2.399)--(3.236,2.453)--(3.328,2.507)%
  --(3.420,2.561)--(3.512,2.615)--(3.604,2.669)--(3.696,2.723)--(3.787,2.777)--(3.879,2.831)%
  --(3.971,2.885)--(4.063,2.938)--(4.155,2.992)--(4.247,3.046)--(4.339,3.100)--(4.431,3.154)%
  --(4.523,3.208)--(4.614,3.262)--(4.706,3.316)--(4.798,3.370)--(4.890,3.423)--(4.982,3.477)%
  --(5.074,3.531)--(5.166,3.585)--(5.258,3.639)--(5.350,3.693)--(5.441,3.747)--(5.533,3.801)%
  --(5.625,3.855)--(5.717,3.909)--(5.809,3.962)--(5.901,4.016)--(5.993,4.070)--(6.085,4.124)%
  --(6.177,4.178)--(6.268,4.232)--(6.360,4.286)--(6.452,4.340)--(6.544,4.394)--(6.636,4.448)%
  --(6.728,4.501)--(6.820,4.555)--(6.912,4.609)--(7.004,4.663)--(7.095,4.717)--(7.187,4.771)%
  --(7.279,4.825)--(7.371,4.879)--(7.463,4.933)--(7.555,4.986)--(7.647,5.040)--(7.739,5.094)%
  --(7.831,5.148)--(7.922,5.202)--(8.014,5.256)--(8.106,5.310)--(8.198,5.364)--(8.290,5.418)%
  --(8.382,5.472)--(8.474,5.525)--(8.566,5.579)--(8.658,5.633)--(8.749,5.687)--(8.841,5.741)%
  --(8.933,5.795)--(9.025,5.849)--(9.117,5.903)--(9.209,5.957)--(9.301,6.011)--(9.393,6.064)%
  --(9.485,6.118)--(9.576,6.172)--(9.668,6.226)--(9.760,6.280)--(9.852,6.334)--(9.944,6.388)%
  --(10.036,6.442)--(10.128,6.496)--(10.220,6.549)--(10.312,6.603)--(10.403,6.657)--(10.495,6.711)%
  --(10.587,6.765)--(10.679,6.819)--(10.771,6.873)--(10.863,6.927)--(10.955,6.981)--(11.047,7.035)%
  --(11.139,7.088)--(11.230,7.142)--(11.322,7.196)--(11.414,7.250)--(11.506,7.304);
\gpcolor{color=gp lt color border}
\node[gp node left] at (8.271,1.319) {Ritorno};
\gpcolor{color=gp lt color 3}
\gpsetlinetype{gp lt plot 3}
\draw[gp path] (10.663,1.319)--(11.579,1.319);
\draw[gp path] (2.409,1.953)--(2.501,2.007)--(2.593,2.061)--(2.685,2.115)--(2.777,2.169)%
  --(2.869,2.223)--(2.960,2.277)--(3.052,2.331)--(3.144,2.385)--(3.236,2.439)--(3.328,2.493)%
  --(3.420,2.547)--(3.512,2.601)--(3.604,2.655)--(3.696,2.709)--(3.787,2.763)--(3.879,2.817)%
  --(3.971,2.871)--(4.063,2.925)--(4.155,2.979)--(4.247,3.033)--(4.339,3.087)--(4.431,3.141)%
  --(4.523,3.195)--(4.614,3.249)--(4.706,3.303)--(4.798,3.357)--(4.890,3.411)--(4.982,3.465)%
  --(5.074,3.519)--(5.166,3.573)--(5.258,3.627)--(5.350,3.681)--(5.441,3.735)--(5.533,3.789)%
  --(5.625,3.843)--(5.717,3.897)--(5.809,3.951)--(5.901,4.005)--(5.993,4.059)--(6.085,4.113)%
  --(6.177,4.167)--(6.268,4.221)--(6.360,4.275)--(6.452,4.329)--(6.544,4.383)--(6.636,4.437)%
  --(6.728,4.491)--(6.820,4.545)--(6.912,4.599)--(7.004,4.653)--(7.095,4.707)--(7.187,4.761)%
  --(7.279,4.815)--(7.371,4.869)--(7.463,4.923)--(7.555,4.977)--(7.647,5.031)--(7.739,5.085)%
  --(7.831,5.139)--(7.922,5.194)--(8.014,5.248)--(8.106,5.302)--(8.198,5.356)--(8.290,5.410)%
  --(8.382,5.464)--(8.474,5.518)--(8.566,5.572)--(8.658,5.626)--(8.749,5.680)--(8.841,5.734)%
  --(8.933,5.788)--(9.025,5.842)--(9.117,5.896)--(9.209,5.950)--(9.301,6.004)--(9.393,6.058)%
  --(9.485,6.112)--(9.576,6.166)--(9.668,6.220)--(9.760,6.274)--(9.852,6.328)--(9.944,6.382)%
  --(10.036,6.436)--(10.128,6.490)--(10.220,6.544)--(10.312,6.598)--(10.403,6.652)--(10.495,6.706)%
  --(10.587,6.760)--(10.679,6.814)--(10.771,6.868)--(10.863,6.922)--(10.955,6.976)--(11.047,7.030)%
  --(11.139,7.084)--(11.230,7.138)--(11.322,7.192)--(11.414,7.246)--(11.506,7.300);
\gpcolor{color=gp lt color border}
\gpsetlinetype{gp lt border}
\draw[gp path] (1.504,7.825)--(1.504,0.985)--(11.947,0.985)--(11.947,7.825)--cycle;
%% coordinates of the plot area
\gpdefrectangularnode{gp plot 1}{\pgfpoint{1.504cm}{0.985cm}}{\pgfpoint{11.947cm}{7.825cm}}
\end{tikzpicture}
%% gnuplot variables

\caption{Spazio!}
\label{fig:spazio1}
\end{figure}

\begin{figure}[H]
    \centering
\begin{tikzpicture}[gnuplot]
%% generated with GNUPLOT 4.6p0 (Lua 5.1; terminal rev. 99, script rev. 100)
%% Tue 25 Mar 2014 05:11:20 PM CET
\path (0.000,0.000) rectangle (12.500,8.750);
\gpcolor{color=gp lt color axes}
\gpsetlinetype{gp lt axes}
\gpsetlinewidth{1.00}
\draw[gp path] (1.504,0.985)--(11.947,0.985);
\gpcolor{color=gp lt color border}
\gpsetlinetype{gp lt border}
\draw[gp path] (1.504,0.985)--(1.684,0.985);
\draw[gp path] (11.947,0.985)--(11.767,0.985);
\node[gp node right] at (1.320,0.985) {-50};
\gpcolor{color=gp lt color axes}
\gpsetlinetype{gp lt axes}
\draw[gp path] (1.504,2.353)--(8.271,2.353);
\draw[gp path] (11.763,2.353)--(11.947,2.353);
\gpcolor{color=gp lt color border}
\gpsetlinetype{gp lt border}
\draw[gp path] (1.504,2.353)--(1.684,2.353);
\draw[gp path] (11.947,2.353)--(11.767,2.353);
\node[gp node right] at (1.320,2.353) { 0};
\gpcolor{color=gp lt color axes}
\gpsetlinetype{gp lt axes}
\draw[gp path] (1.504,3.721)--(11.947,3.721);
\gpcolor{color=gp lt color border}
\gpsetlinetype{gp lt border}
\draw[gp path] (1.504,3.721)--(1.684,3.721);
\draw[gp path] (11.947,3.721)--(11.767,3.721);
\node[gp node right] at (1.320,3.721) { 50};
\gpcolor{color=gp lt color axes}
\gpsetlinetype{gp lt axes}
\draw[gp path] (1.504,5.089)--(11.947,5.089);
\gpcolor{color=gp lt color border}
\gpsetlinetype{gp lt border}
\draw[gp path] (1.504,5.089)--(1.684,5.089);
\draw[gp path] (11.947,5.089)--(11.767,5.089);
\node[gp node right] at (1.320,5.089) { 100};
\gpcolor{color=gp lt color axes}
\gpsetlinetype{gp lt axes}
\draw[gp path] (1.504,6.457)--(11.947,6.457);
\gpcolor{color=gp lt color border}
\gpsetlinetype{gp lt border}
\draw[gp path] (1.504,6.457)--(1.684,6.457);
\draw[gp path] (11.947,6.457)--(11.767,6.457);
\node[gp node right] at (1.320,6.457) { 150};
\gpcolor{color=gp lt color axes}
\gpsetlinetype{gp lt axes}
\draw[gp path] (1.504,7.825)--(11.947,7.825);
\gpcolor{color=gp lt color border}
\gpsetlinetype{gp lt border}
\draw[gp path] (1.504,7.825)--(1.684,7.825);
\draw[gp path] (11.947,7.825)--(11.767,7.825);
\node[gp node right] at (1.320,7.825) { 200};
\gpcolor{color=gp lt color axes}
\gpsetlinetype{gp lt axes}
\draw[gp path] (1.504,0.985)--(1.504,7.825);
\gpcolor{color=gp lt color border}
\gpsetlinetype{gp lt border}
\draw[gp path] (1.504,0.985)--(1.504,1.165);
\draw[gp path] (1.504,7.825)--(1.504,7.645);
\node[gp node center] at (1.504,0.677) { 0};
\gpcolor{color=gp lt color axes}
\gpsetlinetype{gp lt axes}
\draw[gp path] (2.664,0.985)--(2.664,7.825);
\gpcolor{color=gp lt color border}
\gpsetlinetype{gp lt border}
\draw[gp path] (2.664,0.985)--(2.664,1.165);
\draw[gp path] (2.664,7.825)--(2.664,7.645);
\node[gp node center] at (2.664,0.677) { 5};
\gpcolor{color=gp lt color axes}
\gpsetlinetype{gp lt axes}
\draw[gp path] (3.825,0.985)--(3.825,7.825);
\gpcolor{color=gp lt color border}
\gpsetlinetype{gp lt border}
\draw[gp path] (3.825,0.985)--(3.825,1.165);
\draw[gp path] (3.825,7.825)--(3.825,7.645);
\node[gp node center] at (3.825,0.677) { 10};
\gpcolor{color=gp lt color axes}
\gpsetlinetype{gp lt axes}
\draw[gp path] (4.985,0.985)--(4.985,7.825);
\gpcolor{color=gp lt color border}
\gpsetlinetype{gp lt border}
\draw[gp path] (4.985,0.985)--(4.985,1.165);
\draw[gp path] (4.985,7.825)--(4.985,7.645);
\node[gp node center] at (4.985,0.677) { 15};
\gpcolor{color=gp lt color axes}
\gpsetlinetype{gp lt axes}
\draw[gp path] (6.145,0.985)--(6.145,7.825);
\gpcolor{color=gp lt color border}
\gpsetlinetype{gp lt border}
\draw[gp path] (6.145,0.985)--(6.145,1.165);
\draw[gp path] (6.145,7.825)--(6.145,7.645);
\node[gp node center] at (6.145,0.677) { 20};
\gpcolor{color=gp lt color axes}
\gpsetlinetype{gp lt axes}
\draw[gp path] (7.306,0.985)--(7.306,7.825);
\gpcolor{color=gp lt color border}
\gpsetlinetype{gp lt border}
\draw[gp path] (7.306,0.985)--(7.306,1.165);
\draw[gp path] (7.306,7.825)--(7.306,7.645);
\node[gp node center] at (7.306,0.677) { 25};
\gpcolor{color=gp lt color axes}
\gpsetlinetype{gp lt axes}
\draw[gp path] (8.466,0.985)--(8.466,1.165);
\draw[gp path] (8.466,2.397)--(8.466,7.825);
\gpcolor{color=gp lt color border}
\gpsetlinetype{gp lt border}
\draw[gp path] (8.466,0.985)--(8.466,1.165);
\draw[gp path] (8.466,7.825)--(8.466,7.645);
\node[gp node center] at (8.466,0.677) { 30};
\gpcolor{color=gp lt color axes}
\gpsetlinetype{gp lt axes}
\draw[gp path] (9.626,0.985)--(9.626,1.165);
\draw[gp path] (9.626,2.397)--(9.626,7.825);
\gpcolor{color=gp lt color border}
\gpsetlinetype{gp lt border}
\draw[gp path] (9.626,0.985)--(9.626,1.165);
\draw[gp path] (9.626,7.825)--(9.626,7.645);
\node[gp node center] at (9.626,0.677) { 35};
\gpcolor{color=gp lt color axes}
\gpsetlinetype{gp lt axes}
\draw[gp path] (10.787,0.985)--(10.787,1.165);
\draw[gp path] (10.787,2.397)--(10.787,7.825);
\gpcolor{color=gp lt color border}
\gpsetlinetype{gp lt border}
\draw[gp path] (10.787,0.985)--(10.787,1.165);
\draw[gp path] (10.787,7.825)--(10.787,7.645);
\node[gp node center] at (10.787,0.677) { 40};
\gpcolor{color=gp lt color axes}
\gpsetlinetype{gp lt axes}
\draw[gp path] (11.947,0.985)--(11.947,7.825);
\gpcolor{color=gp lt color border}
\gpsetlinetype{gp lt border}
\draw[gp path] (11.947,0.985)--(11.947,1.165);
\draw[gp path] (11.947,7.825)--(11.947,7.645);
\node[gp node center] at (11.947,0.677) { 45};
\draw[gp path] (1.504,7.825)--(1.504,0.985)--(11.947,0.985)--(11.947,7.825)--cycle;
\node[gp node center,rotate=-270] at (0.246,4.405) {Allungamento ($10^{-5} [m]$)};
\node[gp node center] at (6.725,0.215) {Forza (N)};
\node[gp node center] at (6.725,8.287) {Distribuzione dei dati: E16};
\node[gp node left] at (8.271,2.243) {Dati andata};
\gpcolor{color=gp lt color 0}
\gpsetlinetype{gp lt plot 0}
\draw[gp path] (10.663,2.243)--(11.579,2.243);
\draw[gp path] (10.663,2.333)--(10.663,2.153);
\draw[gp path] (11.579,2.333)--(11.579,2.153);
\draw[gp path] (2.319,2.353)--(2.499,2.353);
\draw[gp path] (2.319,2.353)--(2.499,2.353);
\draw[gp path] (3.224,2.900)--(3.404,2.900);
\draw[gp path] (3.224,2.900)--(3.404,2.900);
\draw[gp path] (4.152,3.420)--(4.332,3.420);
\draw[gp path] (4.152,3.420)--(4.332,3.420);
\draw[gp path] (5.057,3.913)--(5.237,3.913);
\draw[gp path] (5.057,3.913)--(5.237,3.913);
\draw[gp path] (5.963,4.432)--(6.143,4.432);
\draw[gp path] (5.963,4.432)--(6.143,4.432);
\draw[gp path] (6.868,4.980)--(7.048,4.980);
\draw[gp path] (6.868,4.980)--(7.048,4.980);
\draw[gp path] (7.773,5.472)--(7.953,5.472);
\draw[gp path] (7.773,5.472)--(7.953,5.472);
\draw[gp path] (8.701,5.965)--(8.881,5.965);
\draw[gp path] (8.701,5.965)--(8.881,5.965);
\draw[gp path] (9.606,6.484)--(9.786,6.484);
\draw[gp path] (9.606,6.484)--(9.786,6.484);
\draw[gp path] (10.511,6.949)--(10.691,6.949);
\draw[gp path] (10.511,6.949)--(10.691,6.949);
\draw[gp path] (3.735,3.532)--(3.915,3.532);
\draw[gp path] (3.735,3.532)--(3.915,3.532);
\gpsetpointsize{4.00}
\gppoint{gp mark 1}{(2.409,2.353)}
\gppoint{gp mark 1}{(3.314,2.900)}
\gppoint{gp mark 1}{(4.242,3.420)}
\gppoint{gp mark 1}{(5.147,3.913)}
\gppoint{gp mark 1}{(6.053,4.432)}
\gppoint{gp mark 1}{(6.958,4.980)}
\gppoint{gp mark 1}{(7.863,5.472)}
\gppoint{gp mark 1}{(8.791,5.965)}
\gppoint{gp mark 1}{(9.696,6.484)}
\gppoint{gp mark 1}{(10.601,6.949)}
\gppoint{gp mark 1}{(3.825,3.532)}
\gppoint{gp mark 1}{(11.121,2.243)}
\gpcolor{color=gp lt color border}
\node[gp node left] at (8.271,1.935) {Dati ritorno};
\gpcolor{color=gp lt color 1}
\gpsetlinetype{gp lt plot 1}
\draw[gp path] (10.663,1.935)--(11.579,1.935);
\draw[gp path] (10.663,2.025)--(10.663,1.845);
\draw[gp path] (11.579,2.025)--(11.579,1.845);
\draw[gp path] (11.416,6.949)--(11.596,6.949);
\draw[gp path] (11.416,6.949)--(11.596,6.949);
\draw[gp path] (10.511,6.457)--(10.691,6.457);
\draw[gp path] (10.511,6.457)--(10.691,6.457);
\draw[gp path] (9.606,5.937)--(9.786,5.937);
\draw[gp path] (9.606,5.937)--(9.786,5.937);
\draw[gp path] (8.701,5.445)--(8.881,5.445);
\draw[gp path] (8.701,5.445)--(8.881,5.445);
\draw[gp path] (7.773,4.952)--(7.953,4.952);
\draw[gp path] (7.773,4.952)--(7.953,4.952);
\draw[gp path] (6.868,4.405)--(7.048,4.405);
\draw[gp path] (6.868,4.405)--(7.048,4.405);
\draw[gp path] (5.963,3.913)--(6.143,3.913);
\draw[gp path] (5.963,3.913)--(6.143,3.913);
\draw[gp path] (5.057,3.420)--(5.237,3.420);
\draw[gp path] (5.057,3.420)--(5.237,3.420);
\draw[gp path] (4.152,2.900)--(4.332,2.900);
\draw[gp path] (4.152,2.900)--(4.332,2.900);
\draw[gp path] (3.224,2.353)--(3.404,2.353);
\draw[gp path] (3.224,2.353)--(3.404,2.353);
\draw[gp path] (3.735,2.460)--(3.915,2.460);
\draw[gp path] (3.735,2.460)--(3.915,2.460);
\gppoint{gp mark 2}{(11.506,6.949)}
\gppoint{gp mark 2}{(10.601,6.457)}
\gppoint{gp mark 2}{(9.696,5.937)}
\gppoint{gp mark 2}{(8.791,5.445)}
\gppoint{gp mark 2}{(7.863,4.952)}
\gppoint{gp mark 2}{(6.958,4.405)}
\gppoint{gp mark 2}{(6.053,3.913)}
\gppoint{gp mark 2}{(5.147,3.420)}
\gppoint{gp mark 2}{(4.242,2.900)}
\gppoint{gp mark 2}{(3.314,2.353)}
\gppoint{gp mark 2}{(3.825,2.460)}
\gppoint{gp mark 2}{(11.121,1.935)}
\gpcolor{color=gp lt color border}
\node[gp node left] at (8.271,1.627) {Andata};
\gpcolor{color=gp lt color 2}
\gpsetlinetype{gp lt plot 2}
\draw[gp path] (10.663,1.627)--(11.579,1.627);
\draw[gp path] (2.409,2.384)--(2.501,2.435)--(2.593,2.487)--(2.685,2.539)--(2.777,2.590)%
  --(2.869,2.642)--(2.960,2.694)--(3.052,2.745)--(3.144,2.797)--(3.236,2.848)--(3.328,2.900)%
  --(3.420,2.952)--(3.512,3.003)--(3.604,3.055)--(3.696,3.107)--(3.787,3.158)--(3.879,3.210)%
  --(3.971,3.262)--(4.063,3.313)--(4.155,3.365)--(4.247,3.416)--(4.339,3.468)--(4.431,3.520)%
  --(4.523,3.571)--(4.614,3.623)--(4.706,3.675)--(4.798,3.726)--(4.890,3.778)--(4.982,3.830)%
  --(5.074,3.881)--(5.166,3.933)--(5.258,3.984)--(5.350,4.036)--(5.441,4.088)--(5.533,4.139)%
  --(5.625,4.191)--(5.717,4.243)--(5.809,4.294)--(5.901,4.346)--(5.993,4.398)--(6.085,4.449)%
  --(6.177,4.501)--(6.268,4.553)--(6.360,4.604)--(6.452,4.656)--(6.544,4.707)--(6.636,4.759)%
  --(6.728,4.811)--(6.820,4.862)--(6.912,4.914)--(7.004,4.966)--(7.095,5.017)--(7.187,5.069)%
  --(7.279,5.121)--(7.371,5.172)--(7.463,5.224)--(7.555,5.275)--(7.647,5.327)--(7.739,5.379)%
  --(7.831,5.430)--(7.922,5.482)--(8.014,5.534)--(8.106,5.585)--(8.198,5.637)--(8.290,5.689)%
  --(8.382,5.740)--(8.474,5.792)--(8.566,5.844)--(8.658,5.895)--(8.749,5.947)--(8.841,5.998)%
  --(8.933,6.050)--(9.025,6.102)--(9.117,6.153)--(9.209,6.205)--(9.301,6.257)--(9.393,6.308)%
  --(9.485,6.360)--(9.576,6.412)--(9.668,6.463)--(9.760,6.515)--(9.852,6.566)--(9.944,6.618)%
  --(10.036,6.670)--(10.128,6.721)--(10.220,6.773)--(10.312,6.825)--(10.403,6.876)--(10.495,6.928)%
  --(10.587,6.980)--(10.679,7.031)--(10.771,7.083)--(10.863,7.134)--(10.955,7.186)--(11.047,7.238)%
  --(11.139,7.289)--(11.230,7.341)--(11.322,7.393)--(11.414,7.444)--(11.506,7.496);
\gpcolor{color=gp lt color border}
\node[gp node left] at (8.271,1.319) {Ritorno};
\gpcolor{color=gp lt color 3}
\gpsetlinetype{gp lt plot 3}
\draw[gp path] (10.663,1.319)--(11.579,1.319);
\draw[gp path] (2.409,1.871)--(2.501,1.922)--(2.593,1.974)--(2.685,2.025)--(2.777,2.076)%
  --(2.869,2.128)--(2.960,2.179)--(3.052,2.231)--(3.144,2.282)--(3.236,2.333)--(3.328,2.385)%
  --(3.420,2.436)--(3.512,2.488)--(3.604,2.539)--(3.696,2.591)--(3.787,2.642)--(3.879,2.693)%
  --(3.971,2.745)--(4.063,2.796)--(4.155,2.848)--(4.247,2.899)--(4.339,2.951)--(4.431,3.002)%
  --(4.523,3.053)--(4.614,3.105)--(4.706,3.156)--(4.798,3.208)--(4.890,3.259)--(4.982,3.310)%
  --(5.074,3.362)--(5.166,3.413)--(5.258,3.465)--(5.350,3.516)--(5.441,3.568)--(5.533,3.619)%
  --(5.625,3.670)--(5.717,3.722)--(5.809,3.773)--(5.901,3.825)--(5.993,3.876)--(6.085,3.928)%
  --(6.177,3.979)--(6.268,4.030)--(6.360,4.082)--(6.452,4.133)--(6.544,4.185)--(6.636,4.236)%
  --(6.728,4.287)--(6.820,4.339)--(6.912,4.390)--(7.004,4.442)--(7.095,4.493)--(7.187,4.545)%
  --(7.279,4.596)--(7.371,4.647)--(7.463,4.699)--(7.555,4.750)--(7.647,4.802)--(7.739,4.853)%
  --(7.831,4.904)--(7.922,4.956)--(8.014,5.007)--(8.106,5.059)--(8.198,5.110)--(8.290,5.162)%
  --(8.382,5.213)--(8.474,5.264)--(8.566,5.316)--(8.658,5.367)--(8.749,5.419)--(8.841,5.470)%
  --(8.933,5.522)--(9.025,5.573)--(9.117,5.624)--(9.209,5.676)--(9.301,5.727)--(9.393,5.779)%
  --(9.485,5.830)--(9.576,5.881)--(9.668,5.933)--(9.760,5.984)--(9.852,6.036)--(9.944,6.087)%
  --(10.036,6.139)--(10.128,6.190)--(10.220,6.241)--(10.312,6.293)--(10.403,6.344)--(10.495,6.396)%
  --(10.587,6.447)--(10.679,6.499)--(10.771,6.550)--(10.863,6.601)--(10.955,6.653)--(11.047,6.704)%
  --(11.139,6.756)--(11.230,6.807)--(11.322,6.858)--(11.414,6.910)--(11.506,6.961);
\gpcolor{color=gp lt color border}
\gpsetlinetype{gp lt border}
\draw[gp path] (1.504,7.825)--(1.504,0.985)--(11.947,0.985)--(11.947,7.825)--cycle;
%% coordinates of the plot area
\gpdefrectangularnode{gp plot 1}{\pgfpoint{1.504cm}{0.985cm}}{\pgfpoint{11.947cm}{7.825cm}}
\end{tikzpicture}
%% gnuplot variables

\caption{Spazio!}
\label{fig:spazio1}
\end{figure}

\begin{figure}[H]
    \centering
\begin{tikzpicture}[gnuplot]
%% generated with GNUPLOT 4.6p0 (Lua 5.1; terminal rev. 99, script rev. 100)
%% Tue 25 Mar 2014 05:11:20 PM CET
\path (0.000,0.000) rectangle (12.500,8.750);
\gpcolor{color=gp lt color axes}
\gpsetlinetype{gp lt axes}
\gpsetlinewidth{1.00}
\draw[gp path] (1.504,0.985)--(11.947,0.985);
\gpcolor{color=gp lt color border}
\gpsetlinetype{gp lt border}
\draw[gp path] (1.504,0.985)--(1.684,0.985);
\draw[gp path] (11.947,0.985)--(11.767,0.985);
\node[gp node right] at (1.320,0.985) {-20};
\gpcolor{color=gp lt color axes}
\gpsetlinetype{gp lt axes}
\draw[gp path] (1.504,1.745)--(8.271,1.745);
\draw[gp path] (11.763,1.745)--(11.947,1.745);
\gpcolor{color=gp lt color border}
\gpsetlinetype{gp lt border}
\draw[gp path] (1.504,1.745)--(1.684,1.745);
\draw[gp path] (11.947,1.745)--(11.767,1.745);
\node[gp node right] at (1.320,1.745) { 0};
\gpcolor{color=gp lt color axes}
\gpsetlinetype{gp lt axes}
\draw[gp path] (1.504,2.505)--(11.947,2.505);
\gpcolor{color=gp lt color border}
\gpsetlinetype{gp lt border}
\draw[gp path] (1.504,2.505)--(1.684,2.505);
\draw[gp path] (11.947,2.505)--(11.767,2.505);
\node[gp node right] at (1.320,2.505) { 20};
\gpcolor{color=gp lt color axes}
\gpsetlinetype{gp lt axes}
\draw[gp path] (1.504,3.265)--(11.947,3.265);
\gpcolor{color=gp lt color border}
\gpsetlinetype{gp lt border}
\draw[gp path] (1.504,3.265)--(1.684,3.265);
\draw[gp path] (11.947,3.265)--(11.767,3.265);
\node[gp node right] at (1.320,3.265) { 40};
\gpcolor{color=gp lt color axes}
\gpsetlinetype{gp lt axes}
\draw[gp path] (1.504,4.025)--(11.947,4.025);
\gpcolor{color=gp lt color border}
\gpsetlinetype{gp lt border}
\draw[gp path] (1.504,4.025)--(1.684,4.025);
\draw[gp path] (11.947,4.025)--(11.767,4.025);
\node[gp node right] at (1.320,4.025) { 60};
\gpcolor{color=gp lt color axes}
\gpsetlinetype{gp lt axes}
\draw[gp path] (1.504,4.785)--(11.947,4.785);
\gpcolor{color=gp lt color border}
\gpsetlinetype{gp lt border}
\draw[gp path] (1.504,4.785)--(1.684,4.785);
\draw[gp path] (11.947,4.785)--(11.767,4.785);
\node[gp node right] at (1.320,4.785) { 80};
\gpcolor{color=gp lt color axes}
\gpsetlinetype{gp lt axes}
\draw[gp path] (1.504,5.545)--(11.947,5.545);
\gpcolor{color=gp lt color border}
\gpsetlinetype{gp lt border}
\draw[gp path] (1.504,5.545)--(1.684,5.545);
\draw[gp path] (11.947,5.545)--(11.767,5.545);
\node[gp node right] at (1.320,5.545) { 100};
\gpcolor{color=gp lt color axes}
\gpsetlinetype{gp lt axes}
\draw[gp path] (1.504,6.305)--(11.947,6.305);
\gpcolor{color=gp lt color border}
\gpsetlinetype{gp lt border}
\draw[gp path] (1.504,6.305)--(1.684,6.305);
\draw[gp path] (11.947,6.305)--(11.767,6.305);
\node[gp node right] at (1.320,6.305) { 120};
\gpcolor{color=gp lt color axes}
\gpsetlinetype{gp lt axes}
\draw[gp path] (1.504,7.065)--(11.947,7.065);
\gpcolor{color=gp lt color border}
\gpsetlinetype{gp lt border}
\draw[gp path] (1.504,7.065)--(1.684,7.065);
\draw[gp path] (11.947,7.065)--(11.767,7.065);
\node[gp node right] at (1.320,7.065) { 140};
\gpcolor{color=gp lt color axes}
\gpsetlinetype{gp lt axes}
\draw[gp path] (1.504,7.825)--(11.947,7.825);
\gpcolor{color=gp lt color border}
\gpsetlinetype{gp lt border}
\draw[gp path] (1.504,7.825)--(1.684,7.825);
\draw[gp path] (11.947,7.825)--(11.767,7.825);
\node[gp node right] at (1.320,7.825) { 160};
\gpcolor{color=gp lt color axes}
\gpsetlinetype{gp lt axes}
\draw[gp path] (1.504,0.985)--(1.504,7.825);
\gpcolor{color=gp lt color border}
\gpsetlinetype{gp lt border}
\draw[gp path] (1.504,0.985)--(1.504,1.165);
\draw[gp path] (1.504,7.825)--(1.504,7.645);
\node[gp node center] at (1.504,0.677) { 0};
\gpcolor{color=gp lt color axes}
\gpsetlinetype{gp lt axes}
\draw[gp path] (2.664,0.985)--(2.664,7.825);
\gpcolor{color=gp lt color border}
\gpsetlinetype{gp lt border}
\draw[gp path] (2.664,0.985)--(2.664,1.165);
\draw[gp path] (2.664,7.825)--(2.664,7.645);
\node[gp node center] at (2.664,0.677) { 5};
\gpcolor{color=gp lt color axes}
\gpsetlinetype{gp lt axes}
\draw[gp path] (3.825,0.985)--(3.825,7.825);
\gpcolor{color=gp lt color border}
\gpsetlinetype{gp lt border}
\draw[gp path] (3.825,0.985)--(3.825,1.165);
\draw[gp path] (3.825,7.825)--(3.825,7.645);
\node[gp node center] at (3.825,0.677) { 10};
\gpcolor{color=gp lt color axes}
\gpsetlinetype{gp lt axes}
\draw[gp path] (4.985,0.985)--(4.985,7.825);
\gpcolor{color=gp lt color border}
\gpsetlinetype{gp lt border}
\draw[gp path] (4.985,0.985)--(4.985,1.165);
\draw[gp path] (4.985,7.825)--(4.985,7.645);
\node[gp node center] at (4.985,0.677) { 15};
\gpcolor{color=gp lt color axes}
\gpsetlinetype{gp lt axes}
\draw[gp path] (6.145,0.985)--(6.145,7.825);
\gpcolor{color=gp lt color border}
\gpsetlinetype{gp lt border}
\draw[gp path] (6.145,0.985)--(6.145,1.165);
\draw[gp path] (6.145,7.825)--(6.145,7.645);
\node[gp node center] at (6.145,0.677) { 20};
\gpcolor{color=gp lt color axes}
\gpsetlinetype{gp lt axes}
\draw[gp path] (7.306,0.985)--(7.306,7.825);
\gpcolor{color=gp lt color border}
\gpsetlinetype{gp lt border}
\draw[gp path] (7.306,0.985)--(7.306,1.165);
\draw[gp path] (7.306,7.825)--(7.306,7.645);
\node[gp node center] at (7.306,0.677) { 25};
\gpcolor{color=gp lt color axes}
\gpsetlinetype{gp lt axes}
\draw[gp path] (8.466,0.985)--(8.466,1.165);
\draw[gp path] (8.466,2.397)--(8.466,7.825);
\gpcolor{color=gp lt color border}
\gpsetlinetype{gp lt border}
\draw[gp path] (8.466,0.985)--(8.466,1.165);
\draw[gp path] (8.466,7.825)--(8.466,7.645);
\node[gp node center] at (8.466,0.677) { 30};
\gpcolor{color=gp lt color axes}
\gpsetlinetype{gp lt axes}
\draw[gp path] (9.626,0.985)--(9.626,1.165);
\draw[gp path] (9.626,2.397)--(9.626,7.825);
\gpcolor{color=gp lt color border}
\gpsetlinetype{gp lt border}
\draw[gp path] (9.626,0.985)--(9.626,1.165);
\draw[gp path] (9.626,7.825)--(9.626,7.645);
\node[gp node center] at (9.626,0.677) { 35};
\gpcolor{color=gp lt color axes}
\gpsetlinetype{gp lt axes}
\draw[gp path] (10.787,0.985)--(10.787,1.165);
\draw[gp path] (10.787,2.397)--(10.787,7.825);
\gpcolor{color=gp lt color border}
\gpsetlinetype{gp lt border}
\draw[gp path] (10.787,0.985)--(10.787,1.165);
\draw[gp path] (10.787,7.825)--(10.787,7.645);
\node[gp node center] at (10.787,0.677) { 40};
\gpcolor{color=gp lt color axes}
\gpsetlinetype{gp lt axes}
\draw[gp path] (11.947,0.985)--(11.947,7.825);
\gpcolor{color=gp lt color border}
\gpsetlinetype{gp lt border}
\draw[gp path] (11.947,0.985)--(11.947,1.165);
\draw[gp path] (11.947,7.825)--(11.947,7.645);
\node[gp node center] at (11.947,0.677) { 45};
\draw[gp path] (1.504,7.825)--(1.504,0.985)--(11.947,0.985)--(11.947,7.825)--cycle;
\node[gp node center,rotate=-270] at (0.246,4.405) {Allungamento ($10^{-5} [m]$)};
\node[gp node center] at (6.725,0.215) {Forza (N)};
\node[gp node center] at (6.725,8.287) {Distribuzione dei dati: E17};
\node[gp node left] at (8.271,2.243) {Dati andata};
\gpcolor{color=gp lt color 0}
\gpsetlinetype{gp lt plot 0}
\draw[gp path] (10.663,2.243)--(11.579,2.243);
\draw[gp path] (10.663,2.333)--(10.663,2.153);
\draw[gp path] (11.579,2.333)--(11.579,2.153);
\draw[gp path] (2.319,1.745)--(2.499,1.745);
\draw[gp path] (2.319,1.745)--(2.499,1.745);
\draw[gp path] (3.224,2.353)--(3.404,2.353);
\draw[gp path] (3.224,2.353)--(3.404,2.353);
\draw[gp path] (4.152,2.961)--(4.332,2.961);
\draw[gp path] (4.152,2.961)--(4.332,2.961);
\draw[gp path] (5.057,3.531)--(5.237,3.531);
\draw[gp path] (5.057,3.531)--(5.237,3.531);
\draw[gp path] (5.963,4.101)--(6.143,4.101);
\draw[gp path] (5.963,4.101)--(6.143,4.101);
\draw[gp path] (6.868,4.709)--(7.048,4.709);
\draw[gp path] (6.868,4.709)--(7.048,4.709);
\draw[gp path] (7.773,5.260)--(7.953,5.260);
\draw[gp path] (7.773,5.260)--(7.953,5.260);
\draw[gp path] (8.701,5.849)--(8.881,5.849);
\draw[gp path] (8.701,5.849)--(8.881,5.849);
\draw[gp path] (9.606,6.419)--(9.786,6.419);
\draw[gp path] (9.606,6.419)--(9.786,6.419);
\draw[gp path] (10.511,7.027)--(10.691,7.027);
\draw[gp path] (10.511,7.027)--(10.691,7.027);
\draw[gp path] (11.416,7.597)--(11.596,7.597);
\draw[gp path] (11.416,7.597)--(11.596,7.597);
\gpsetpointsize{4.00}
\gppoint{gp mark 1}{(2.409,1.745)}
\gppoint{gp mark 1}{(3.314,2.353)}
\gppoint{gp mark 1}{(4.242,2.961)}
\gppoint{gp mark 1}{(5.147,3.531)}
\gppoint{gp mark 1}{(6.053,4.101)}
\gppoint{gp mark 1}{(6.958,4.709)}
\gppoint{gp mark 1}{(7.863,5.260)}
\gppoint{gp mark 1}{(8.791,5.849)}
\gppoint{gp mark 1}{(9.696,6.419)}
\gppoint{gp mark 1}{(10.601,7.027)}
\gppoint{gp mark 1}{(11.506,7.597)}
\gppoint{gp mark 1}{(11.121,2.243)}
\gpcolor{color=gp lt color border}
\node[gp node left] at (8.271,1.935) {Dati ritorno};
\gpcolor{color=gp lt color 1}
\gpsetlinetype{gp lt plot 1}
\draw[gp path] (10.663,1.935)--(11.579,1.935);
\draw[gp path] (10.663,2.025)--(10.663,1.845);
\draw[gp path] (11.579,2.025)--(11.579,1.845);
\draw[gp path] (11.416,7.597)--(11.596,7.597);
\draw[gp path] (11.416,7.597)--(11.596,7.597);
\draw[gp path] (10.511,6.989)--(10.691,6.989);
\draw[gp path] (10.511,6.989)--(10.691,6.989);
\draw[gp path] (9.606,6.400)--(9.786,6.400);
\draw[gp path] (9.606,6.400)--(9.786,6.400);
\draw[gp path] (8.701,5.811)--(8.881,5.811);
\draw[gp path] (8.701,5.811)--(8.881,5.811);
\draw[gp path] (7.773,5.241)--(7.953,5.241);
\draw[gp path] (7.773,5.241)--(7.953,5.241);
\draw[gp path] (6.868,4.633)--(7.048,4.633);
\draw[gp path] (6.868,4.633)--(7.048,4.633);
\draw[gp path] (5.963,4.063)--(6.143,4.063);
\draw[gp path] (5.963,4.063)--(6.143,4.063);
\draw[gp path] (5.057,3.455)--(5.237,3.455);
\draw[gp path] (5.057,3.455)--(5.237,3.455);
\draw[gp path] (4.152,2.923)--(4.332,2.923);
\draw[gp path] (4.152,2.923)--(4.332,2.923);
\draw[gp path] (3.224,2.315)--(3.404,2.315);
\draw[gp path] (3.224,2.315)--(3.404,2.315);
\draw[gp path] (2.319,1.745)--(2.499,1.745);
\draw[gp path] (2.319,1.745)--(2.499,1.745);
\gppoint{gp mark 2}{(11.506,7.597)}
\gppoint{gp mark 2}{(10.601,6.989)}
\gppoint{gp mark 2}{(9.696,6.400)}
\gppoint{gp mark 2}{(8.791,5.811)}
\gppoint{gp mark 2}{(7.863,5.241)}
\gppoint{gp mark 2}{(6.958,4.633)}
\gppoint{gp mark 2}{(6.053,4.063)}
\gppoint{gp mark 2}{(5.147,3.455)}
\gppoint{gp mark 2}{(4.242,2.923)}
\gppoint{gp mark 2}{(3.314,2.315)}
\gppoint{gp mark 2}{(2.409,1.745)}
\gppoint{gp mark 2}{(11.121,1.935)}
\gpcolor{color=gp lt color border}
\node[gp node left] at (8.271,1.627) {Andata};
\gpcolor{color=gp lt color 2}
\gpsetlinetype{gp lt plot 2}
\draw[gp path] (10.663,1.627)--(11.579,1.627);
\draw[gp path] (2.409,1.770)--(2.501,1.829)--(2.593,1.887)--(2.685,1.946)--(2.777,2.005)%
  --(2.869,2.064)--(2.960,2.123)--(3.052,2.182)--(3.144,2.241)--(3.236,2.300)--(3.328,2.358)%
  --(3.420,2.417)--(3.512,2.476)--(3.604,2.535)--(3.696,2.594)--(3.787,2.653)--(3.879,2.712)%
  --(3.971,2.771)--(4.063,2.829)--(4.155,2.888)--(4.247,2.947)--(4.339,3.006)--(4.431,3.065)%
  --(4.523,3.124)--(4.614,3.183)--(4.706,3.242)--(4.798,3.300)--(4.890,3.359)--(4.982,3.418)%
  --(5.074,3.477)--(5.166,3.536)--(5.258,3.595)--(5.350,3.654)--(5.441,3.712)--(5.533,3.771)%
  --(5.625,3.830)--(5.717,3.889)--(5.809,3.948)--(5.901,4.007)--(5.993,4.066)--(6.085,4.125)%
  --(6.177,4.183)--(6.268,4.242)--(6.360,4.301)--(6.452,4.360)--(6.544,4.419)--(6.636,4.478)%
  --(6.728,4.537)--(6.820,4.596)--(6.912,4.654)--(7.004,4.713)--(7.095,4.772)--(7.187,4.831)%
  --(7.279,4.890)--(7.371,4.949)--(7.463,5.008)--(7.555,5.067)--(7.647,5.125)--(7.739,5.184)%
  --(7.831,5.243)--(7.922,5.302)--(8.014,5.361)--(8.106,5.420)--(8.198,5.479)--(8.290,5.537)%
  --(8.382,5.596)--(8.474,5.655)--(8.566,5.714)--(8.658,5.773)--(8.749,5.832)--(8.841,5.891)%
  --(8.933,5.950)--(9.025,6.008)--(9.117,6.067)--(9.209,6.126)--(9.301,6.185)--(9.393,6.244)%
  --(9.485,6.303)--(9.576,6.362)--(9.668,6.421)--(9.760,6.479)--(9.852,6.538)--(9.944,6.597)%
  --(10.036,6.656)--(10.128,6.715)--(10.220,6.774)--(10.312,6.833)--(10.403,6.891)--(10.495,6.950)%
  --(10.587,7.009)--(10.679,7.068)--(10.771,7.127)--(10.863,7.186)--(10.955,7.245)--(11.047,7.304)%
  --(11.139,7.362)--(11.230,7.421)--(11.322,7.480)--(11.414,7.539)--(11.506,7.598);
\gpcolor{color=gp lt color border}
\node[gp node left] at (8.271,1.319) {Ritorno};
\gpcolor{color=gp lt color 3}
\gpsetlinetype{gp lt plot 3}
\draw[gp path] (10.663,1.319)--(11.579,1.319);
\draw[gp path] (2.409,1.728)--(2.501,1.787)--(2.593,1.846)--(2.685,1.905)--(2.777,1.964)%
  --(2.869,2.023)--(2.960,2.082)--(3.052,2.141)--(3.144,2.200)--(3.236,2.259)--(3.328,2.318)%
  --(3.420,2.377)--(3.512,2.436)--(3.604,2.495)--(3.696,2.554)--(3.787,2.613)--(3.879,2.672)%
  --(3.971,2.731)--(4.063,2.790)--(4.155,2.849)--(4.247,2.909)--(4.339,2.968)--(4.431,3.027)%
  --(4.523,3.086)--(4.614,3.145)--(4.706,3.204)--(4.798,3.263)--(4.890,3.322)--(4.982,3.381)%
  --(5.074,3.440)--(5.166,3.499)--(5.258,3.558)--(5.350,3.617)--(5.441,3.676)--(5.533,3.735)%
  --(5.625,3.794)--(5.717,3.853)--(5.809,3.912)--(5.901,3.971)--(5.993,4.030)--(6.085,4.089)%
  --(6.177,4.148)--(6.268,4.207)--(6.360,4.266)--(6.452,4.325)--(6.544,4.384)--(6.636,4.443)%
  --(6.728,4.502)--(6.820,4.561)--(6.912,4.620)--(7.004,4.679)--(7.095,4.738)--(7.187,4.797)%
  --(7.279,4.856)--(7.371,4.915)--(7.463,4.974)--(7.555,5.033)--(7.647,5.092)--(7.739,5.151)%
  --(7.831,5.210)--(7.922,5.269)--(8.014,5.328)--(8.106,5.387)--(8.198,5.446)--(8.290,5.505)%
  --(8.382,5.564)--(8.474,5.623)--(8.566,5.682)--(8.658,5.741)--(8.749,5.800)--(8.841,5.859)%
  --(8.933,5.918)--(9.025,5.977)--(9.117,6.036)--(9.209,6.095)--(9.301,6.154)--(9.393,6.213)%
  --(9.485,6.272)--(9.576,6.331)--(9.668,6.390)--(9.760,6.449)--(9.852,6.508)--(9.944,6.567)%
  --(10.036,6.626)--(10.128,6.685)--(10.220,6.744)--(10.312,6.803)--(10.403,6.862)--(10.495,6.921)%
  --(10.587,6.980)--(10.679,7.039)--(10.771,7.098)--(10.863,7.157)--(10.955,7.216)--(11.047,7.275)%
  --(11.139,7.334)--(11.230,7.393)--(11.322,7.452)--(11.414,7.511)--(11.506,7.570);
\gpcolor{color=gp lt color border}
\gpsetlinetype{gp lt border}
\draw[gp path] (1.504,7.825)--(1.504,0.985)--(11.947,0.985)--(11.947,7.825)--cycle;
%% coordinates of the plot area
\gpdefrectangularnode{gp plot 1}{\pgfpoint{1.504cm}{0.985cm}}{\pgfpoint{11.947cm}{7.825cm}}
\end{tikzpicture}
%% gnuplot variables

\caption{Spazio!}
\label{fig:spazio1}
\end{figure}


\section{Discussione dati sperimentali}
Compatibilmente con le previsioni teoriche, si è verificato che l'accelerazione della slitta non dipende dalla sua massa, infatti, aggiungendo il peso, varia soltanto la velocità, e che, diminuendo la velocità iniziale della slitta con l'aggiunta dello spessore, la decelerazione data dalla forza di attrito dell'aria è rimasta invariata.

\section{Conclusioni}
Si ha una migliore stima di \textbf{g}

% Esempio di inclusione di un immagine ./img/spazio1.png etichettata fig:spazio, al centro, larga 0.8 per larghezza del testo, con testo sotto Spazio!
\section {Grafici}
Di seguito sono riportati i grafici delle velocità medie e le rette interpolanti.
\subsection {}
%\include{}

\section{Codice}	

\paragraph{}
Riportiamo in seguito il programma utilizzato per l'elaborazione dei dati
%Verbatim non interpreta l'imput lasciando il testo com'è: ideale per inserire codice
\begin{verbatim}

\end{verbatim}


%\subsection{Esempio immagini}
%\begin{figure}[p]
% \centering
% \includegraphics[width=0.8\textwidth]{spazio1}
% \caption{Spazio!}
% \label{fig:spazio1}
%\end{figure}

\end{document}
